\documentclass{beamer}
\usepackage{asymptote}
\usepackage[french]{babel}
\usepackage[utf8]{inputenc}
\usepackage[T1]{fontenc}
\usepackage{lmodern}
\usepackage{enumerate}
\usepackage{amsmath, amssymb, amsthm}
% for use in integrals and derivatives
\newcommand{\dd}{\mathrm d}
\usepackage{hyperref}
\usetheme{Warsaw}
\usefonttheme{professionalfonts}
\setbeamertemplate{navigation symbols}{}
\setbeamertemplate{headline}{}

\title{Le modèle de Thomas-Fermi-von Weizs\"acker ultra-relativiste}
\author{Gaspard Jankowiak et Antoine Levitt}
\date{24 mars 2011}
\institute{Groupe de travail des thésards en mécanique quantique}

\renewcommand{\leq}{\leqslant}
\renewcommand{\le}{\leqslant}
\renewcommand{\geq}{\geqslant}
\renewcommand{\ge}{\geqslant}

\begin{document}
\AtBeginSection[]
{
  \begin{frame}
    \tableofcontents[currentsection]
  \end{frame}
}

% \newcommand{\OutlineColumnNumbers}{2}

\frame{\titlepage}
% \frame{\tableofcontents}
% \frame{
%   \begin{columns}
%     \begin{column}{0.5\textwidth}
%       \tableofcontents[sections={1-2}]
%     \end{column}
%     \begin{column}{0.5\textwidth}
%       \tableofcontents[sections={3-4}]
%     \end{column}
%   \end{columns}
% }
\section{Le modèle}
\subsection{Présentation du modèle}
\frame{
  \frametitle{Le modèle URTFW}
  \begin{align*}
    \mathcal E(\rho) &= \underbrace{a^2 \int (\nabla \rho^{1/3})^2 \dd x + b^2
      \int \rho^{4/3} \dd x}_{\text{énergie cinétique}} \\
    &-
    \underbrace{\int V(x) \rho(x) \dd x}_\text{attraction des noyaux}\\
    &+ \underbrace{D(\rho,\rho)}_\text{répulsion électronique}
  \end{align*}
  \begin{itemize}
  \item $V(x) = \alpha \sum_i \frac{z_i}{|x - R_i|}$
  \item $D(\rho,\rho) = \frac \alpha 2 \int \frac{\rho(x) \rho(y)}{|x-y|} \dd x \dd y
    \ge 0$
  \end{itemize}
}
\frame{
  \frametitle{Justification}
  \begin{itemize}
  \item Modélise le gaz d'électrons de densité $\rho$ autour d'un
    atome ou d'une molécule
  \item Modèle quantique avec corrections relativistes
  \item Limite d'un modèle de QED (Engel-Dreizler)
  \item Pas très physique, modèle jouet
  \end{itemize}
}
\frame{
  \frametitle{Questions mathématiques}
  \begin{itemize}
  \item Stabilité ?
  \item Si $\inf \mathcal E(\rho) = - \infty$, alors système instable
  \item Cas atomique $R_1 = 0$. Changement d'échelle : $\sigma(x) = \lambda^{3}
    \rho(\lambda x)$: zoom d'un facteur $\lambda$ sur le noyau, en préservant la charge
    totale $\int \sigma = \int \rho$
  \item $ \mathcal E(\sigma) = \lambda \mathcal
    E(\rho)$
  \item Seulement deux possibilités
    \begin{enumerate}
    \item $\exists \rho, \mathcal E(\rho) < 0$ : $\inf \mathcal E = -
      \infty$ (instable)
    \item $\forall \rho, \mathcal E(\rho) \ge 0$ : $\inf \mathcal E =
      0$ (stable)
    \end{enumerate}
  \item Instabilité: $\int \frac{\rho(x)}{|x|} \dd x$ domine, $\lambda
    \rightarrow +\infty$, concentration
    $\rho \rightarrow \delta_{R_i}$
  \item Stabilité: pas d'échelle naturelle (au contraire de
    Schr\"odinger pour H)
  \end{itemize}
}
\subsection{Présentation des papiers}
\frame{
  \frametitle{Premier papier}
  \begin{itemize}
  \item Benguria - Pérez - Oyarzun: The ultrarelativistic TFW model (2002)
  \item Cas atomique. Résultat : $z < z_s \implies$ stabilité, $z >
    z_i \implies$ instabilité.
  \item Technique de preuve : perturbation sur l'interaction
    inter-électronique, réduction au cas radial par réarrangement
    symétrique, et contrôle de l'attraction des noyaux
  \item Extension au cas moléculaire par réarrangement symétrique :
    tous les noyaux au même endroit, pire cas, borne grossière sur
    $\sum_i z_i$
  \end{itemize}
}
\frame{
  \frametitle{Deuxième papier}
  \begin{itemize}
  \item Benguria - Loss - Siedentop: Stability of atoms and molecules
    in an ultrarelativistic TFW model (2008)
  \item Cas moléculaire avec prise en compte de la répulsion des noyaux
  \item Résultat : borne sur $\max_i z_i$
  \item Technique de preuve : contrôle sur $x$ proche de $R_i$, et $x$
    loin de $R_i$, via diagramme de Voronoï
  \end{itemize}
}
\section{Cas atomique}
\subsection{Principe d'incertitude}
\frame{
  \frametitle{Principe d'incertitude}
  \begin{itemize}
  \item Sous-problème :
    \begin{equation*}
      F(\psi) = \frac{a^2 \int (\nabla \psi)^2 \dd x + b^2
        \int \psi^{4} \dd x}{
        \alpha z \int \frac{\psi^3(x)}{|x|} \dd x}
    \end{equation*}
  \item Équivalent au problème de départ en négligeant l'interaction électronique
  \item Raisonable parce que faible physiquement, et pas indispensable
    mathématiquement
  \item Beaucoup plus simple à résoudre par réduction au cas radial
  \item Lemme : $F(\psi) \ge F_0 = \frac{4 a b}{3 \alpha z}$
  \item ``Principe d'incertitude'': si les électrons sont localisés
    autour de 0 (dénominateur grand), alors grande énergie cinétique (numérateur grand)
  \end{itemize}
}
\frame{
  \frametitle{Preuve du lemme : réduction au cas radial}
  \begin{itemize}
  \item Regarder $F$ sous $\psi \rightarrow \psi^*$, où $\psi^*$ est la symétrisée de
    Schwartz (ref: Lieb \& Loss, Analysis)
  \item $\psi^*(x)$ est radial, décroissant
  \item Propriétés
    \begin{enumerate}
    \item $\int (\nabla \psi^*)^2 \le \int (\nabla \psi)^2$ (lissage
      des oscillations)
    \item $\int (\psi^*)^4 = \int \psi^4$ (préserve les normes $L^p$)
    \item $\int \psi^3 \frac 1 {|x|} \le \int (\psi^*)^3 (\frac 1
      {|x|})^* = \int (\psi^*)^3 \frac 1 {|x|}$ ($\int |f g| \le \int
      |f^* g^*|$ : penser à supp $f $ disjoint de supp $g$)
    \end{enumerate}
  \item $F(\psi^*) \le F(\psi)$
  \item Regarder $F$ sur des fonctions radiales
  \end{itemize}
}

\frame{
  \frametitle{Preuve du lemme : cas radial}
  \begin{align*}
    F(\psi) &= \frac{a^2 \int (\psi')^2 r^2 \dd r + b^2
      \int \psi^{4} r^2 \dd r}{
      \alpha z \int {\psi^3} r \dd r}\\
     &\overset{\text{IPP}}{=} \frac {4 a b }{3 \alpha z}\;\frac{ \int \overbrace{(a\psi')^2}^{x^2} r^2 \dd r +
      \int \overbrace{(b \psi^{2})^2}^{y^2} r^2 \dd r}{
      \int - 2  \underbrace{a \psi'}_x\underbrace{b \psi^2}_{y} r^2
      \dd r}
  \end{align*}
  \begin{itemize}
  \item $- 2 x y \le x^2 + y^2$
  \item Conclusion: $F(\psi) \ge F_0 = \frac{4 a b}{3 \alpha z}$
  \item Bonus : on remonte le cas d'égalité $x = - y$, on en déduit une équation
    différentielle $a \psi' = - b \psi^2$ qu'on résoud:
    \begin{equation*}
      \psi(r) = \psi_{R}(r) = \frac a b \frac {1}{r+R}
    \end{equation*}

  \end{itemize}
}
\subsection{Étude de $\mathcal{E}$}
\frame{
  \frametitle{Fonctionelle complète}
    \begin{align*}
    \mathcal E(\rho) &= {a^2 \int (\nabla \rho^{1/3})^2 \dd x + b^2
      \int \rho^{4/3} \dd x}
    -
    \alpha z {\int \frac{ \rho(x)}{|x|} \dd x}
    +{D(\rho,\rho)}\\
    &= \left[\alpha z \int \frac \rho {|x|} \dd x\right] \left[F(\rho)
      - 1 + \frac{D(\rho, \rho)}{\alpha z \int \frac \rho {|x|} \dd x} \right]
  \end{align*}
  \begin{itemize}
  \item Si $F_0 > 1$, alors $\mathcal E(\rho) \geq 0$ : stabilité
  \item Si $F_0 < 1$, alors l'argument ne permet pas de conclure
  \item Il est naturel de tester $\mathcal E$ sur $\psi_R$, qui
    minimise déjà $F$. Si $\mathcal E(\psi_R) < 0$, alors instabilité
  \end{itemize}
}
\frame{
  \frametitle{Conclusion}
  \begin{align*}
    \inf \mathcal E (\rho) =
    \begin{cases}
      0 & \text{pour } z < \frac{4 a b}{3 \alpha}\\
      -\infty & \text{pour } z > \frac{4 a b}{3 \alpha} + \frac{7 \pi
        a^3}{6 b^3}
    \end{cases}
  \end{align*}
  \begin{itemize}
  \item Facile de montrer l'existence d'un $z_c$ critique
  \item On a donc $\frac{4 a b}{3 \alpha} \le z_c \le \frac{4 a b}{3 \alpha} + \frac{7 \pi
        a^3}{6 b^3}$. Valeur exacte inconnue
    \item Pour valeurs physiques, $\frac{4 a b}{3 \alpha} \approx 70$,
      $\frac{7 \pi
        a^3}{6 b^3} \approx 0.2$
    \item L'encadrement est bon. L'interaction entre les électrons est
      faible, et l'approximation du lemme était physiquement raisonable
  \end{itemize}
}
\subsection{Extension au cas moléculaire}
\frame{
  \frametitle{Cas moléculaire}
  \begin{itemize}
  \item On peut prouver la stabilité pour $\sum_i z_i < \frac{4 a b}{3
      \alpha}$ de même que précédemment, parce que $(\sum_i
    \frac{z_i}{|x-R_i|})^* = (\sum_i z_i) \frac 1 {|x|}$
  \item Physiquement, correspond à mettre tous les noyaux au même
    endroit : pire cas
  \item Pas de preuve d'instabilité
  \item Pas physique : existence de longues molécules (chimie
    organique)
  \item Stabilité fournie par l'énergie de répulsion des noyaux $U =
    \sum_{i < j} \frac{z_i z_j}{|R_i - R_j|}$ qui empêche le pire cas
  \item Nécessité de méthodes spécifiques pour prouver la stabilité de
    $\mathcal E + U$ en prenant en compte la géométrie de la molécule
  \end{itemize}
}
\section{Cas moléculaire}
\begin{frame}{Résultat}
    \begin{itemize}
        \item Pour traiter la géométrie du problème on minimise
            \[\mathcal{E} + U.\]
        \item Stabilité, i.e. $\mathcal{E}(\rho)+ U\geq0,$ si
          \[ z_i \leq \frac{4ab}{3\alpha}\sqrt{1-x} = z_{cs}
          \sqrt{1-x},\] où $x$ est solution d'un polynôme du 3ième
          degré. $x \approx 0.04$, même borne que le cas atomique avec
          correction $\approx$ 2\%.
    \end{itemize}
\end{frame}
\subsection{Géométrie}
\begin{frame}{Charges identiques}
            \[\mathcal{E} + \alpha\sum_{i < j} \frac{z_i z_j}{|R_i-R_j|}.\]
    \begin{itemize}
        \item L'énergie est concave en $z_i \in [0, z]$\\
            $\Rightarrow$ minimum sur les $z_i$ atteint en $0$ ou $z$.
        \item Si minimum atteint en $0$ alors on a un noyau de moins.
        \item On considère donc toutes les charges égales à $z$.
    \end{itemize}
\end{frame}
\begin{frame}{Diagramme de Voronoï}
    On définit
    \begin{align*}
      \Gamma_j &= \left\{x \; \big| \; \forall\, k\neq j, |x-R_j| < |x-R_k|\right\},\\
      D_j &= \mathrm{dist}(R_j, \mathit{\partial}\Gamma_k),\\
      B_j &= {B(R_j, D_j)}.
    \end{align*}
    \begin{center}
    \includegraphics[width=0.6\textwidth]{voronoi.pdf}
    \end{center}
\end{frame}
\begin{frame}{Séparation du potentiel}{Action longue distance}
    On définit le potentiel « longue distance », qui, à l'intérieur d'une cellule, ne « voit pas
    » le noyau associé. Dans $\Gamma_i$,
    \[\Phi(x) = \alpha z \sum_{j\neq i} \frac{1}{|x-R_j|}.\]
\end{frame}
\subsection{Séparation des cellules}
\newcommand{\vv}{\widetilde{V}}
\newcommand{\vvv}{\widehat{V}}
\begin{frame}{Séparation des cellules}
    On sépare l'énergie potentielle :
    \[V = \vv + \Phi,\]
    avec
    \[\vv(x) = \frac z {|x - R_i|} \text{ dans $\Gamma_i$}\]
    \begin{align*}
    \mathcal{E} + U =&
    \overbrace{a^2\int (\nabla \rho^\frac{1}{3})^2 + b^2 \int \rho^{4/3}
    - \int \vv \rho}^{\mathcal{E}_1}\\
    &\overbrace{-\int \Phi \rho +  \frac{\alpha}{2}\int\frac{\rho(x) \rho(y)}{|x-y|}+U,}^{\mathcal{E}_2}
    \end{align*}
\end{frame}

\begin{frame}{Inégalité électrostatique de Lieb et Yau}

    Contrôle de l'énergie potentielle longue distance
    par l'interaction électronique et l'interaction nucléaire.

    \vspace{1cm}

    Pour toute densité $\rho$,
    \[\int\Phi(x)\rho(x) \leq \frac{\alpha}{2}\iint
    \frac{\rho(x)\rho(y)}{|x-y|} + U -
    \alpha \frac{z^2}{8}\sum_j \frac{1}{D_j}.\]
\end{frame}

\begin{frame}{Interaction inter-cellulaire}
  \begin{align*}
    {\mathcal{E}_2} = {-\int \Phi \rho +  \frac{\alpha}{2}\int\frac{\rho(x) \rho(y)}{|x-y|}+U,}
  \end{align*}
  On contrôle $\int \Phi \rho$ avec l'inégalité électrostatique
  \begin{align*}
        \mathcal{E}_2 \ge& \frac{\alpha z^2}{8}\sum_j \frac{1}{D_j}.
  \end{align*}

\end{frame}
\begin{frame}{Interaction intra-cellulaire}
  \begin{align*}
    \mathcal E_1 = {a^2\int (\nabla \rho^\frac{1}{3})^2 + b^2 \int \rho^{4/3}
    - \int \vv \rho}
\end{align*}

Contrôle de $\int \vv \rho$ ?

Si $\vv$ n'était pas singulier, on pourrait utiliser H\"older
  \begin{align*}
    \int \vv \rho \leq \Vert\vv\Vert_{4} \Vert\rho\Vert_{4/3}
  \end{align*}
  pour faire apparaître $(\int \rho^{4/3})^{3/4}$ et conclure.

  Problème: $\vv \not \in L^{4}$.

  On sépare la partie singulière

\end{frame}

\subsection{Isolation de la singularité}
\begin{frame}{Isolation de la singularité}
  \begin{align*}
    \mathcal E_1 = {a^2\int (\nabla \rho^\frac{1}{3})^2 + b^2 \int \rho^{4/3}
    - \int \vv \rho}
\end{align*}
  Contrôle de $\int \vv \rho$

  \begin{align*}
    \int \vv \rho &= \sum_j \underbrace{\int_{B_j} \vv
      \rho}_{\text{?}} + \sum_j \underbrace{\int_{\Gamma_j -
      B_j} \vv \rho}_{\text{Hölder}}
\end{align*}
\end{frame}

\definecolor{gris}{rgb}{.2 .2 .2}
\begin{frame}{Principe d'incertitude modifié}
  Contrôle de la singularité coulombienne dans toute boule (et non pas
  sur tout l'espace comme dans le cas atomique)

    \vspace{1cm}

    Pour toute densité $\rho$ :
    \begin{align*}
      \frac {4ab}{3} \int_{B_j}
       \frac 1 {|x|}\rho(x)\dd x &\leq
      a^2\int_{B_j}\left|\nabla \rho(x)^{\tiny{\frac{1}{3}}}\right|^2 \dd x + b^2 \int_{B_j} \rho(x)^{4/3}\dd x \\&+ \frac{2ab}{R} \int_{B_j}
    \rho(x)\dd x.
    \end{align*}

\end{frame}


\begin{frame}{Contrôle de la singularité}
On contrôle $\int_{B_j} \vv \rho$ par le P.I.M. si
\begin{align*}
  \alpha z \leq \frac{4ab}{3}
\end{align*}

Mais on veut garder du $\int \rho^{4/3}$ pour $\int_{\Gamma_j - B_j}
\vv \rho$:
\begin{align*}
  b^2 &= b_1^2 + b_2^2,\\
  \alpha z &= \frac{4ab_2}{3}
\end{align*}
\end{frame}
\begin{frame}[fragile]{Contrôle de la partie non singulière}
  \begin{align*}
    \mathcal{E}_1&\geq
    b_1^2 \int \rho^{4/3} - \sum_j \int_{B_j}\frac{3\alpha z}{2 D_j} \rho
    - \sum_j\int_{\scriptscriptstyle\Gamma_j\setminus B_j} \vv \rho\\
    &\ge
    b_1^2 \int \rho^{4/3} - \int_{\mathbb{R}^3} \vvv\rho,
  \end{align*}
  où

  \begin{minipage}{0.2\linewidth}
    \centering
    \small
    \begin{align*}
      \vvv(x) &= \left\{\begin{aligned}
          \frac{3\alpha z}{2D_j} && x \in B_j\\
          \frac{\alpha z}{|x-R_j|} && x \in \Gamma_j \setminus B_j\end{aligned}\right.
    \end{align*}
  \end{minipage}
  \begin{minipage}{0.5\linewidth}
    \centering
    \begin{asy}
import graph;
size(5cm);
real i=0.5;
real f(real x) {return 1/abs(x);}
real g(real x) {return 1/i+1;}
draw(graph(f,i,4),blue+3*linewidth(currentpen));
draw(graph(g,0,i),blue+3*linewidth(currentpen));
xaxis('$|x-R_j|$');
yaxis('$\widehat{V}$');
labelx('$D_j$',i);
draw((i,g(i))--(i,0),dashed);
\end{asy}
\end{minipage}

$\vvv$ n'est plus singulier, on peut le contrôler par Hölder.
\end{frame}

\begin{frame}{Contrôle de la partie non singulière (2)}
            \begin{align*}
                \int \vvv \rho \le& \Vert\vvv\Vert_4 \Vert\rho\Vert_{4/3},\\
                \intertext{donc}
                \mathcal{E}_1 \ge&\,b_1^2 \int \rho^{4/3} -
                \Vert\vvv\Vert_4 \left(\int \rho^{4/3}\right)^{3/4}\\
                \ge& b_1^2 X - \Vert\vvv\Vert_4 X^{3/4}
                \intertext{En optimisant sur $X$ on a :}
                \mathcal{E}_1 \ge& -\frac{1}{4}\left( \frac{3}{4b_1^2}
                \right)^3\Vert \vvv \Vert_4^4.
              \end{align*}
              On calcule
              \begin{align*}
                \int \vvv^4 \leq \sum_j \frac 1 {D_j} (\alpha z)^4 (\underbrace{\frac 4 3 \pi}_{B_j}
                  +  \underbrace{{3 \pi}}_{\mathbb R^3 \setminus B_j})
              \end{align*}
\end{frame}
\subsection{Conclusion}
\begin{frame}{Conclusion}
  Finalement, si $b_2(z) < b$,
  \[\mathcal{E} + U \ge M(z) \sum_j \frac{1}{D_j}\]

  On en déduit une condition sur $z$ et le résultat
        \[\mathcal{E} + U\ge 0\quad\text { si }\quad \max z_i \le \frac{4ab}{3\alpha}\sqrt{1-x},\]
        \medskip
        où $x$ est solution de
        \[\frac{1-x}{x^3}=\frac{b^4}{a^2}\left( \frac{4}{3}
        \right)^2\frac{1}{2\pi\alpha(4+9\alpha^4)}.\]

\end{frame}
\section{Conclusion}
\begin{frame}{Conclusion générale}
  \begin{itemize}
  \item Cas atomique : preuve par réarrangement symétrique
  \item Cas moléculaire : condition de stabilité sur $\max z_i$
  \item On se ramène à un cas quasi-atomique dans les cellules du
    diagramme de Voronoï
  \item Résultat presque aussi bon que dans le cas atomique
  \item On peut montrer le même résultat d'instabilité que dans le cas
    atomique en concentrant $\rho$ sur un atome
  \end{itemize}
\end{frame}
\begin{frame}{Points pour discussion et questions ouvertes}
  \begin{itemize}
  \item Origine du modèle
  \item Argument de concavité
  \item $z_c$
  \item ...
  \end{itemize}
\end{frame}
\end{document}
