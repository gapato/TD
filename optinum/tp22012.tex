\documentclass[10pt,a4paper,fleqn]{report}
\usepackage{a4wide}
\usepackage{amsmath,amssymb}
\usepackage[french]{babel}
\usepackage[utf8]{inputenc}
\usepackage[T1]{fontenc}
\usepackage{lmodern}
\usepackage{oldstyle}
\usepackage{url}
%%%%%%%%%%%%%%%%%%%%%%%%%%%%%%%%%%%%%%%%%%%%%%%%%%%%%%%%%%%%
\usepackage{GLtd}
\renewcommand{\thesection}{ \textbf{\Large \arabic{section}.}}
%%%%%%%%%%%%%%%%%%%%%%%%%%%%%%%%%%%%%%%%%%%%%%%%%%%%%%%%%%%\`u\`u
\setlength{\oddsidemargin}{1pt}
\setlength{\evensidemargin}{1pt}
\setlength{\textheight}{590pt}
\setlength{\textwidth}{470pt}

\makeatletter
\def\cleardoublepage{\clearpage\if@twoside\ifodd\c@page\else\hbox{}\thispagestyle{empty}\newpage\fi\fi}
\makeatother

\newcommand{\matlab}{\textsc{Matlab}}

%%%%%%%%%%%%%%%%%%%%%%%%%%%%%%%%%%%%%%%%%%%%%%%%%%%%%%%%%%%%

\newcommand{\R}{\mathbb R}
\newcommand{\onit}{\begin{enumerate}}
\newcommand{\offit}{\end{enumerate}}
\newcommand{\grad}{\nabla}
\newcommand{\hess}{\nabla^2}
\newcommand{\on}{\begin{displaymath}}
\newcommand{\off}{\end{displaymath}}
\renewcommand{\P}{\mathcal P}
\newcommand{\push}[1]{\rule{0pt}{0pt}\hspace{#1pt}}
\renewcommand{\footnote}{\stepcounter{footnote}\raisebox{5 pt}{\scriptsize(\thefootnote)}\footnotetext}

%%%%%%%%%%%%%%%%%%%%%%%%%%%%%%%%%%%%%%%%%%%%%%%%%%%%%%%%%%%%

\begin{document}

\cleardoublepage

\noindent
Universit\'e Paris-Dauphine     \hfill      Optimisation num\'erique\\
Licence maths applis      \hfill      Ann\'ee \textos{2011}/\textos{2012}

\medskip

\hrule

\medskip



\begin{center}

\textbf{\huge TP 2}

\smallskip

\rule{10cm}{0.4pt}

\end{center}

On résoud le problème
\begin{align*}
  \text{min} \;\;\;\;&x^{T} A x\\
  \text{s.c.}\;\;\;\;&\begin{cases}
    &e^{T} x = 1\\
  &r^{T} x = \rho\\
  &x \geq 0
  \end{cases}
\end{align*}

par la méthode du gradient projeté à pas fixe. Les fichiers à
télécharger sont disponibles sur

\verb+http://www.ceremade.dauphine.fr/~levitt/optinum/+

\begin{enumerate}
\item Coder les fonctions de l'exercice 4, et les tester.
\item Télécharger le fichier \verb+gradfix.m+, et l'utiliser comme
  modèle pour une fonction \verb+gradproj+ qui accepte en paramètre
  une fonction \verb+proj+ chargée de réaliser la projection. Que se
  passerait-il si on avait utilisé la norme du gradient comme critère
  d'arrêt ?
\item Comment choisir le pas de la méthode de gradient ? Le point
  initial ?
\item Tester votre algorithme avec $A =
  \begin{pmatrix}
    4&0&0\\0&64&0\\0&0&100
  \end{pmatrix}, b =
  \begin{pmatrix}
    5\\10\\15
  \end{pmatrix}$. Étudier l'impact de $\rho$ et $\mu$.
\item Modifier la procédure \verb+gradproj+ pour qu'elle retourne la
  liste des itérés $x_{n}$. Tracer la courbe de convergence $||x_{n+1}
  - x_{n}||$ en fonction de $n$.
\item L'analyse théorique de l'algorithme de gradient mène à un
  résultat de convergence linéaire du type
  \begin{align*}
    ||x_{n} - x^{*}|| \leq K \nu^{n}
  \end{align*}
  où $K > 0$, et $\nu < 1$ est la constante de convergence.

  Combien d'itérations faut-il pour obtenir une précision de
  $10^{-16}$ sur $x$ ?
  
  Prouver que
  \begin{align*}
    ||x_{n+1} - x_{n}|| \leq K' \nu^{n}
  \end{align*}

  Montrer qu'un tracé en échelle semi-logarithmique permet de vérifier
  visuellement cette inégalité. Se documenter sur les possibilités de
  régression linéaire en Matlab pour obtenir numériquement la
  constante $\nu$.
\item Comment la constante de convergence $\nu$ varie-t-elle en
  fonction du pas $t$ ?
\item Faire varier le pas t, goldensearch, autre jeu de données ...
\end{enumerate}
\end{document}
