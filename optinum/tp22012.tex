\documentclass[10pt,a4paper,fleqn]{report}
\usepackage{a4wide}
\usepackage{amsmath,amssymb}
\usepackage[french]{babel}
\usepackage[utf8]{inputenc}
\usepackage[T1]{fontenc}
\usepackage{lmodern}
\usepackage{oldstyle}
\usepackage{hyperref}
%%%%%%%%%%%%%%%%%%%%%%%%%%%%%%%%%%%%%%%%%%%%%%%%%%%%%%%%%%%%
\usepackage{GLtd}
\renewcommand{\thesection}{ \textbf{\Large \arabic{section}.}}
%%%%%%%%%%%%%%%%%%%%%%%%%%%%%%%%%%%%%%%%%%%%%%%%%%%%%%%%%%%\`u\`u
\setlength{\oddsidemargin}{1pt}
\setlength{\evensidemargin}{1pt}
\setlength{\textheight}{590pt}
\setlength{\textwidth}{470pt}

\makeatletter
\def\cleardoublepage{\clearpage\if@twoside\ifodd\c@page\else\hbox{}\thispagestyle{empty}\newpage\fi\fi}
\makeatother

\newcommand{\matlab}{\textsc{Matlab}}

%%%%%%%%%%%%%%%%%%%%%%%%%%%%%%%%%%%%%%%%%%%%%%%%%%%%%%%%%%%%

\newcommand{\R}{\mathbb R}
\newcommand{\onit}{\begin{enumerate}}
\newcommand{\offit}{\end{enumerate}}
\newcommand{\grad}{\nabla}
\newcommand{\hess}{\nabla^2}
\newcommand{\on}{\begin{displaymath}}
\newcommand{\off}{\end{displaymath}}
\renewcommand{\P}{\mathcal P}
\newcommand{\push}[1]{\rule{0pt}{0pt}\hspace{#1pt}}
\renewcommand{\footnote}{\stepcounter{footnote}\raisebox{5 pt}{\scriptsize(\thefootnote)}\footnotetext}

%%%%%%%%%%%%%%%%%%%%%%%%%%%%%%%%%%%%%%%%%%%%%%%%%%%%%%%%%%%%

\begin{document}

\cleardoublepage

\noindent
Universit\'e Paris-Dauphine     \hfill      Optimisation num\'erique\\
Licence maths applis      \hfill      Ann\'ee \textos{2011}/\textos{2012}

\medskip

\hrule

\medskip



\begin{center}

\textbf{\huge TP 2}

\smallskip

\rule{10cm}{0.4pt}

\end{center}

On résoud le problème
\begin{align*}
  \text{min} \quad&F(x)=x^{T} A x\\
  \text{s.c.}\quad&\begin{cases}
    &e^{T} x = 1\\
  &r^{T} x = \rho\\
  &x \geq 0
  \end{cases}
\end{align*}

par la méthode du gradient projeté à pas fixe. Les fichiers à
télécharger sont disponibles sur

\url{http://www.ceremade.dauphine.fr/~levitt/optinum/}

\begin{enumerate}
\item Coder les fonctions \verb+oracle_tp2+ et \verb+proj_tp2+
    de l'exercice 5, et les tester.
\item Télécharger le fichier \verb+gradfix.m+, et l'utiliser comme
  modèle pour une fonction \verb+gradproj+ qui accepte en paramètre
  une fonction \verb+proj+ chargée de réaliser la projection.
  Peut-on utiliser la norme du gradient comme critère d'arrêt ? Pourquoi ?
\item Comment choisir le pas de la méthode de gradient ? Le point
  initial ?
\item Tester votre algorithme avec
    \[A =
  \begin{pmatrix}
    4&0&0\\0&64&0\\0&0&100
  \end{pmatrix},\;r =
  \begin{pmatrix}
    5\\10\\15
  \end{pmatrix}.\]
  Pour ceci, télécharger le fichier de données \verb+data.m+ (à placer dans votre répertoire
  courant), où sont définis 3 jeux de données, la matrice $A$ et le vecteur $r$ ci-dessus
  correspondent au premier. Voici un exemple d'utilisation :
  \begin{verbatim}
  ...
  global A r e rho mu
  data                      % initialisation de la structure stocks
  A = stocks(1).covar       % on utilise le premier jeu de données...
  r = stocks(1).returns
  ...
  \end{verbatim}

  Étudier l'impact de $\rho$ et $\mu$. Idéalement, comment devrait-on choisir $\mu$, le paramètre de
  pénalisation ?
\item Modifier la procédure \verb+gradproj+ pour qu'elle retourne la
  liste des itérés $x_{n}$ et tracer la courbe de convergence
  $\left\Vert x_{n+1} - x_{n}\right\Vert$ en fonction de $n$.
\item L'analyse théorique de l'algorithme de gradient mène à un
  résultat de convergence linéaire du type
  \begin{align*}
    \left\Vert x_{n} - x^{*}\right\Vert \leq K \nu^{n}
  \end{align*}
  où $K > 0$, et $\nu < 1$ est la constante de convergence.

  Combien d'itérations faut-il pour obtenir une précision de
  $10^{-16}$ sur $x$ ?

  Prouver que
  \begin{align*}
    \left\Vert x_{n+1} - x_{n}\right\Vert \leq K' \nu^{n}
  \end{align*}

  Montrer qu'un tracé en échelle semi-logarithmique permet de vérifier
  visuellement cette inégalité, et en utilisant la fonction \verb+polyfit+ de Matlab,
  obtenir num\'eriquement une majoration de la constante $\nu$.

\item Comment la constante de convergence $\nu$ varie-t-elle en
  fonction du pas $t$ ?
\item Recommencer l'analyse de la question de 4 avec les 2 autres jeux de données, et tracer les
    courbes de convergence. Qu'est-ce qui change ?
\item On souhaite maintenant utiliser une autre méthode que la descente à pas fixe. Pour cela, on va
    utiliser un algorithme de recherche linéaire qui est une variante de la méthode de dichotomie :
    ``golden section search''\footnote{Introduite en 1953 par Keifer :
    \url{http://en.wikipedia.org/wiki/Golden_section_search}}.

    Télécharger le fichier \verb+goldensearch.m+ qui définie une procédure prenant en paramètre une
    fonction $f:\mathbb{R}\rightarrow\mathbb{R}$ et une tolérance $\delta$, et qui rend la valeur approchée
    du minimum de cette fonction dans l'intervalle $[0,\infty)$.

    Écrire une fonction \verb+gradproj_gold+ (basée la fonction \verb+gradproj+)
    qui utilise \verb+goldensearch+.

    \emph{Indication} : on peut définir une fonction ``inline'' en Matlab en utilisant la syntaxe suivante :
    \begin{verbatim}
        oracle_1d = @(t) oracle_tp2(x-t*g)
    \end{verbatim}
    ce qui définit \verb+oracle_1d+ qui prend en paramètre \verb+t+.
\item Tracer les nouvelles courbes de convergence.
\item Modifier les fonctions \verb+goldensearch+ et \verb+gardproj_gold+ pour cette dernière retourne le nombre total d'appels à l'oracle,
    et le comparer au nombre d'appels pour le gradient à pas fixe.
\end{enumerate}
\end{document}
