\documentclass[12pt,a4paper,fleqn]{report}
\usepackage{a4wide}
\usepackage{amsmath,amssymb}
\usepackage[french]{babel}
\usepackage[utf8]{inputenc}
\usepackage[T1]{fontenc}
\usepackage{lmodern}
\usepackage{oldstyle}
\usepackage{url}
%%%%%%%%%%%%%%%%%%%%%%%%%%%%%%%%%%%%%%%%%%%%%%%%%%%%%%%%%%%%
\usepackage{GLtd}
\renewcommand{\thesection}{ \textbf{\Large \arabic{section}.}}
%%%%%%%%%%%%%%%%%%%%%%%%%%%%%%%%%%%%%%%%%%%%%%%%%%%%%%%%%%%\`u\`u
\setlength{\oddsidemargin}{1pt}
\setlength{\evensidemargin}{1pt}
\setlength{\textheight}{590pt}
\setlength{\textwidth}{470pt}

\makeatletter
\def\cleardoublepage{\clearpage\if@twoside\ifodd\c@page\else\hbox{}\thispagestyle{empty}\newpage\fi\fi}
\makeatother

\newcommand{\matlab}{\textsc{Matlab}}

%%%%%%%%%%%%%%%%%%%%%%%%%%%%%%%%%%%%%%%%%%%%%%%%%%%%%%%%%%%%

\newcommand{\R}{\mathbb R}
\newcommand{\onit}{\begin{enumerate}}
\newcommand{\offit}{\end{enumerate}}
\newcommand{\grad}{\nabla}
\newcommand{\hess}{\nabla^2}
\newcommand{\on}{\begin{displaymath}}
\newcommand{\off}{\end{displaymath}}
\renewcommand{\P}{\mathcal P}
\newcommand{\push}[1]{\rule{0pt}{0pt}\hspace{#1pt}}
\renewcommand{\footnote}{\stepcounter{footnote}\raisebox{5 pt}{\scriptsize(\thefootnote)}\footnotetext}

%%%%%%%%%%%%%%%%%%%%%%%%%%%%%%%%%%%%%%%%%%%%%%%%%%%%%%%%%%%%

\begin{document}

\cleardoublepage

\noindent
Universit\'e Paris-Dauphine     \hfill      Optimisation num\'erique\\
Licence maths applis      \hfill      Ann\'ee \textos{2011}/\textos{2012}

\medskip

\hrule

\medskip



\begin{center}

\textbf{\huge TD1}

\smallskip

\rule{10cm}{0.4pt}

\end{center}



\begin{exercice}
  On considère une fonction $F : \R^{N} \to \R$ de classe $C^{2}$.
  \begin{questions}
  \item Rappeler la formule de Taylor à l'ordre deux autour d'un point $x$.
  \item Montrer que si $x$ est un minimum local de $F$, alors $\nabla
    F(x) = 0$.
  \item Montrer que si $x$ est un minimum local de $F$, alors
    $\nabla^{2} F(x)$ est symétrique semi-définie positive.
  \item Montrer que si $x$ vérifie $\nabla F(x) = 0$, $\nabla^{2}
    F(x)$ définie positive, alors $x$ est un minimum local de $F$.
  \item Donner un exemple où $\nabla^{2} F(x)$ est semi-définie
    positive, mais $x$ n'est pas un minimum local.
  \end{questions}
\end{exercice}

\begin{exercice}
  Soit $\phi(t) = F(x+th)$, où $x$ et $h \in \R^{N}$ sont
  fixes. Calculer $\phi'$ et $\phi''$. Si on cherche à minimiser $F$
  à partir du point $x$, dans quelles directions peut-on se déplacer
  pour être assuré de faire descendre la valeur de $F$ avec un pas $t$
  suffisamment petit ?
\end{exercice}

\begin{exercice}
  Soit $F(x) = \frac 1 2 x^{T} A x - b^{T} x$, où $b \in \R^{N}$ et $A
  \in M_{N}(\R)$ est inversible.
  \begin{questions}
  \item Calculer le gradient et la Hessienne de $F$.
  \item Quels sont les points critiques de $F$ ? À quelle condition
    sont-ils des minima locaux ? Globaux ?
  \end{questions}
\end{exercice}

\begin{exercice}
  Soit $F : M_{N}(\R) \to \R$ définie par $F(M) = \text{tr}(M^{3} - A
  M)$, où $A \in M_{N}(R)$ est une matrice fixée.
  \begin{questions}
  \item Rappeler le produit scalaire canonique de $M_{N}(\R)$
  \item Calculer le gradient et la Hessienne de $F$.
  \end{questions}
\end{exercice}

\begin{exercice}
  À périmètre fixe, quel est le rectangle ayant l'aire maximale ?
  L'aire minimale ? (utiliser deux méthodes différentes)
\end{exercice}

\begin{exercice}
  Un maître-nageur de plage court à 20 km/h, et nage à 10 km/h. En
  partant de la plage, quelle trajectoire doit-il choisir pour sauver
  un baigneur en train de se noyer?
\end{exercice}

\end{document}
