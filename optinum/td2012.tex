\documentclass[12pt,a4paper,fleqn]{report}
\usepackage{a4wide}
\usepackage{amsmath,amssymb}
\usepackage[french]{babel}
\usepackage[utf8]{inputenc}
\usepackage[T1]{fontenc}
\usepackage{lmodern}
\usepackage{oldstyle}
\usepackage{url}
%%%%%%%%%%%%%%%%%%%%%%%%%%%%%%%%%%%%%%%%%%%%%%%%%%%%%%%%%%%%
\usepackage{GLtd}
\renewcommand{\thesection}{ \textbf{\Large \arabic{section}.}}
%%%%%%%%%%%%%%%%%%%%%%%%%%%%%%%%%%%%%%%%%%%%%%%%%%%%%%%%%%%\`u\`u
\setlength{\oddsidemargin}{1pt}
\setlength{\evensidemargin}{1pt}
\setlength{\textheight}{590pt}
\setlength{\textwidth}{470pt}

\makeatletter
\def\cleardoublepage{\clearpage\if@twoside\ifodd\c@page\else\hbox{}\thispagestyle{empty}\newpage\fi\fi}
\makeatother

\newcommand{\matlab}{\textsc{Matlab}}

%%%%%%%%%%%%%%%%%%%%%%%%%%%%%%%%%%%%%%%%%%%%%%%%%%%%%%%%%%%%

\newcommand{\R}{\mathbb R}
\newcommand{\onit}{\begin{enumerate}}
\newcommand{\offit}{\end{enumerate}}
\newcommand{\grad}{\nabla}
\newcommand{\hess}{\nabla^2}
\newcommand{\on}{\begin{displaymath}}
\newcommand{\off}{\end{displaymath}}
\renewcommand{\P}{\mathcal P}
\newcommand{\push}[1]{\rule{0pt}{0pt}\hspace{#1pt}}
\renewcommand{\footnote}{\stepcounter{footnote}\raisebox{5 pt}{\scriptsize(\thefootnote)}\footnotetext}

%%%%%%%%%%%%%%%%%%%%%%%%%%%%%%%%%%%%%%%%%%%%%%%%%%%%%%%%%%%%

\begin{document}

\cleardoublepage

\noindent
Universit\'e Paris-Dauphine     \hfill      Optimisation num\'erique\\
Licence maths applis      \hfill      Ann\'ee \textos{2011}/\textos{2012}

\medskip

\hrule

\medskip



\begin{center}

\textbf{\huge TD/TP 1}

\smallskip

\rule{10cm}{0.4pt}

\end{center}


\section{Rappels de calcul différentiel}
\begin{exercice}
  On considère une fonction $F : \R^{N} \to \R$ de classe $C^{2}$.
  \begin{questions}
  \item Rappeler la formule de Taylor à l'ordre deux autour d'un point $x$.
  \item Montrer que si $x$ est un minimum local de $F$, alors $\nabla
    F(x) = 0$.
  \item Montrer que si $x$ est un minimum local de $F$, alors
    $\nabla^{2} F(x)$ est symétrique semi-définie positive.
  \item Montrer que si $x$ vérifie $\nabla F(x) = 0$, $\nabla^{2}
    F(x)$ définie positive, alors $x$ est un minimum local de $F$.
  \item Donner un exemple où $\nabla^{2} F(x)$ est semi-définie
    positive, mais $x$ n'est pas un minimum local.
  \end{questions}
\end{exercice}

\begin{exercice}
  Soit $\phi(t) = F(x+th)$, où $x$ et $h \in \R^{N}$ sont
  fixes. Calculer $\phi'$ et $\phi''$. Si on cherche à minimiser $F$
  à partir du point $x$, dans quelles directions peut-on se déplacer
  pour être assuré de faire descendre la valeur de $F$ avec un pas $t$
  suffisamment petit ?
\end{exercice}

\section{Calcul de gradients et de Hessiennes}

\begin{exercice}
  Soit $F(x) = \frac 1 2 x^{T} A x - b^{T} x$, où $b \in \R^{N}$ et $A
  \in M_{N}(\R)$ est inversible.
  \begin{questions}
  \item Pourquoi peut-on supposer sans perte de généralité que $A$ est
    symétrique ?
  \item Calculer le gradient et la Hessienne de $F$.
  \item Quels sont les points critiques de $F$ ? À quelle condition
    sont-ils des minima locaux ? Globaux ?
  \end{questions}
\end{exercice}

\begin{exercice}
  Soit $F : \R^{N} \to \R$ définie par $F(x) = ||{A x - b}||^{2}$, où
  $b \in \R^{M}$ et $A \in M_{M \times N}(\R)$, avec $M > N$.
  \begin{questions}
  \item Calculer le gradient et la hessienne de $F$.
  \item À quelle condition le problème d'optimisation de $F$ est-il
    bien posé ? Comment calculer le minimiseur en pratique ?
  \end{questions}
\end{exercice}

\begin{exercice}
  Soit $F : M_{N}(\R) \to \R$ définie par $F(M) = \text{tr}(M^{3} - A
  M)$, où $A \in M_{N}(R)$ est une matrice fixée.
  \begin{questions}
  \item Rappeler le produit scalaire canonique de $M_{N}(\R)$
  \item Calculer le gradient et la Hessienne de $F$.
  \end{questions}
\end{exercice}

\section{Problèmes d'optimisation}

\begin{exercice}
  À périmètre fixe, quel est le rectangle ayant l'aire maximale ?
  L'aire minimale ? (utiliser deux méthodes différentes)
\end{exercice}

\begin{exercice}
  Un maître-nageur de plage court à 20 km/h, et nage à 10
  km/h. Formuler un problème d'optimisation pour trouver la
  trajectoire optimale qu'il doit choisir pour sauver un baigneur en
  train de se noyer.
\end{exercice}

\section{Implémentation Matlab}
Guide de survie Matlab
\begin{itemize}
\item Chaque fonction doit être placée dans un fichier portant le même
  nom
\item \verb+help eye+ affiche l'aide pour la fonction
  \verb+eye+. \verb+doc eye+ l'affiche dans le navigateur d'aide.
\item Préférer le style vectoriel au style scalaire : on écrira
\begin{verbatim}
A = B + C;
\end{verbatim} plutôt que
\begin{verbatim}
for i = 1:N
    for j = 1:N
        A(i,j) = B(i,j) + C(i,j);
    end
end
\end{verbatim}

  Afin d'éviter les boucles, on peut utiliser les commandes d'accès
  par plages de valeurs : par exemple,
\begin{verbatim}
A(:,1) = A(:,1) + A(2,:)'
\end{verbatim}
 sur une matrice carrée ajoute la deuxième ligne à la
  première colonne
\item Utiliser \verb+;+ à la fin d'une ligne pour ne pas afficher le
  résultat de la commande.
\end{itemize}

\begin{exercice}
  Un oracle est une fonction qui à $x$ renvoie $F(x)$, et
  éventuellement $\nabla F(x)$ et $\nabla^{2} F(x)$. Par exemple,
  l'oracle de $F(x) = ||x||^{2}$ peut s'écrire
\begin{verbatim}
function [F,grad,hess] = oracle(x)
    F = x'*x;
    grad = 2*x;
    hess = eye(numel(x));
end
\end{verbatim}

Cette fonction est à mettre dans le fichier \verb+oracle.m+ et à
appeler par une commande du type
\begin{verbatim}
    [F,grad,hess] = oracle([1;2]);
\end{verbatim}

Ecrire un oracle pour les fonctions des exercices 3, 4 et 5. On
supposera que les données $A$ et $b$ sont des variables globales,
qu'on déclarera comme telles. Tester l'oracle de l'exercice 3 avec les
données $A = \begin{pmatrix}2&1&0\\1&3&1\\0&1&2\end{pmatrix}, b
= \begin{pmatrix}1\\2\\3\end{pmatrix}$.
\end{exercice}

\begin{exercice}
  Ecrire une fonction \verb+grad_findiff+ qui calcule une
  approximation du gradient d'une fonction $F$ à un point $x$ par
  différences finies. Comment choisir le pas de différence finie en
  pratique ? Que se passe-t-il si ce pas est trop petit ? Trop grand ?
  Vérifier le résultat de cette fonction sur l'oracle de l'exercice 3.
\end{exercice}

\begin{exercice}
  On se propose de minimiser la fonction $F$ de l'exercice 3 par un
  algorithme de gradient à pas fixe.
  \begin{questions}
  \item Comment vérifier que la matrice proposée dans l'exercice 8 est
    bien définie positive ?
  \item Comment choisir le pas ? Que se passe-t-il si ce pas est trop
    petit ? Trop grand ?
  \item Proposer plusieurs critères d'arrêt pour l'algorithme. Lequel
    choisissez-vous ?
  \item Ecrire une procédure générique \verb+gradfix+ qui prend en
    paramètre un oracle, un point de départ et un pas, et renvoie la
    solution du problème de minimisation. La tester sur l'oracle de
    l'exercice 3, et comparer avec la solution exacte.
  \end{questions}

\end{exercice}

\begin{exercice}
  Se donner des positions de départ pour le maître nageur, la mer et
  le baigneur, et trouver la trajectoire optimale grâce à la procédure
  \verb+gradfix+.

  Optionnel : visualiser le résultat en Matlab.
\end{exercice}


\end{document}
