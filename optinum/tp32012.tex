\documentclass[10pt,a4paper,fleqn]{report}
\usepackage{a4wide}
\usepackage{amsmath,amssymb}
\usepackage[french]{babel}
\usepackage[utf8]{inputenc}
\usepackage[T1]{fontenc}
\usepackage{lmodern}
\usepackage{oldstyle}
\usepackage{url}
%%%%%%%%%%%%%%%%%%%%%%%%%%%%%%%%%%%%%%%%%%%%%%%%%%%%%%%%%%%%
\renewcommand{\thesection}{ \textbf{\Large \arabic{section}.}}
%%%%%%%%%%%%%%%%%%%%%%%%%%%%%%%%%%%%%%%%%%%%%%%%%%%%%%%%%%%\`{u}\`{u}
\setlength{\oddsidemargin}{1pt}
\setlength{\evensidemargin}{1pt}
\setlength{\textheight}{590pt}
\setlength{\textwidth}{470pt}

\makeatletter
\def\cleardoublepage{\clearpage\if@twoside\ifodd\c@page\else\hbox{}\thispagestyle{empty}\newpage\fi\fi}
\makeatother
\DeclareMathOperator*{\argmin}{arg\,min}
\newcommand{\matlab}{\textsc{Matlab}}

%%%%%%%%%%%%%%%%%%%%%%%%%%%%%%%%%%%%%%%%%%%%%%%%%%%%%%%%%%%%
\newcommand{\dotp}[2]{\langle #1 , #2\rangle}
\newcommand{\R}{\mathbb R}
\newcommand{\onit}{\begin{enumerate}}
\newcommand{\offit}{\end{enumerate}}
\newcommand{\grad}{\nabla}
\newcommand{\hess}{\nabla^2}
\newcommand{\on}{\begin{displaymath}}
\newcommand{\off}{\end{displaymath}}
\renewcommand{\P}{\mathcal P}
\newcommand{\push}[1]{\rule{0pt}{0pt}\hspace{#1pt}}
\renewcommand{\footnote}{\stepcounter{footnote}\raisebox{5 pt}{\scriptsize(\thefootnote)}\footnotetext}
\renewcommand{\tt}{\texttt}
\newtheorem{question}{Question}
%%%%%%%%%%%%%%%%%%%%%%%%%%%%%%%%%%%%%%%%%%%%%%%%%%%%%%%%%%%%

\begin{document}

\cleardoublepage

\noindent
Universit\'e Paris-Dauphine     \hfill      Optimisation num\'erique\\
D\'epartement MIDO              \hfill      François-Xavier Vialard\\
DUMI2E troisi\`eme ann\'ee      \hfill      ann\'ee \textos{2011}/\textos{2012}

\medskip

\hrule

\medskip

\begin{center}

\textbf{\huge S\'{e}ance de TD/TP n\textsuperscript{o}3}

\smallskip

\rule{10cm}{0.4pt}

\end{center}


\section*{Objectif}
On s'int\'eresse au probl\`eme de d\'ebruitage d'une image donn\'ee par des m\'ethodes variationnelles. La donn\'ee d'une image num\'erique est une matrice de taille $(n,m)$. On utilisera l'image test classique en traitement d'image \texttt{lena.jpg} de taille $(512,512)$ pour valider les algorithmes impl\'ement\'es. Les m\'ethodes de d\'ebruitage reposent souvent sur des m\'ethodes variationnelles qui passent par une repr\'esentation continue de l'image. C'est \`a dire que les mod\`eles sont introduits sur des espaces fonctionnels et une approximation discr\`ete est ensuite utilis\'ee pour les tests num\'eriques.


\section{Mod\'elisation}

On consid\`ere qu'une image est la representation discr\`ete d'une fonction $f:[0,a]\times[0,b] \mapsto \R^2$ qui est donn\'ee par une matrice $[f_{i,j}]_{(i,j)\in [1,n]\times[1,m]}$.
On propose le premier mod\`ele suivant: l'image recherch\'ee appartient \`a l'espace des fonctions d\'erivables et dont la d\'eriv\'ee est dans $L^2([0,a]\times[0,b],\R^2)$. On suppose que la donn\'ee est une fonction $g \in L^2$ qui est la donn\'ee bruit\'ee (le manque de r\'egularit\'e de la fonction $g$ est du \`a la pr\'esence de bruit). A priori, on peut \'ecrire:
\begin{equation}
g = f+ \varepsilon 
\end{equation}
avec $\varepsilon \in L^2([0,a]\times[0,b],\R) $.

 On formule maintenant un probl\`eme d'optimisation permettant de trouver un candidat not\'e $\hat{f}$ pour repr\'esenter la fonction $f$:
\begin{equation}
\hat{f} = \argmin \frac12 \| u -  g\|_{L^2}^2 + \frac{1}{2\beta}\| \nabla u \|^2_{L^2}\,,
\end{equation}
o\`u $\beta$ est un param\`etre r\'eel positif.
Ce probl\`eme d'optimisation propose donc de trouver une fonction $\hat{f}$ qui va donner un bon compromis entre la donn\'ee (du fait du terme $ \| u -  g\|_{L^2}^2 $) et l'hypoth\`ese de mod\'elisation $\nabla u \in L^2$ (du fait du terme de droite $\| \nabla u \|^2_{L^2}$).

\begin{question}
\'Ecrire une version discr\`ete des termes $\frac12 \| u -  g\|_{L^2}^2$ et $\frac{1}{2\beta}\| \nabla u \|^2_{L^2}$ qui soient des fonctions quadratiques des valeurs $[u_{i,j}]_{(i,j)\in [1,n]\times[1,m]}$ et $[g_{i,j}]_{(i,j)\in [1,n]\times[1,m]}$.
\end{question}

\begin{question}
Quelle est la r\'egularit\'e de la fonctionnelle en fonction des coefficients matriciels?
Discuter de l'existence et de l'unicit\'e de solutions du probl\`eme discret.
\end{question}

\begin{question}
Calculer le gradient de la fonctionnelle en fonction des coefficients de la matrice $u$ et de la matrice $g$.
\end{question}

\begin{question}
Impl\'ementer l'algorithme du gradient conjugu\'e pour r\'esoudre num\'eriquement le probl\`eme discretis\'e. On rappelle l'algorithme pour minimiser $\frac12 \dotp{Ax}{x}+\dotp{b}{x}$:

Initialisation de $k=0$,
$x_0 \in \R^{nm}$,  $r_0 \leftarrow Ax_0 + b$ and $p_0 \leftarrow -r_0$ et it\'eration tant que $r_k\neq 0$ de
\begin{equation}
\begin{cases}
    \alpha_k \leftarrow \frac{\dotp{r_k}{r_k} }{ \dotp{p_k}{A p_k} } \\
    x_{k+1} \leftarrow x_k + \alpha_k p_k         \\
    r_{k+1} \leftarrow r_k + \alpha_k A p_k       \\
    \beta_{k+1} \leftarrow \frac{ \dotp{r_{k+1}}{r_{k+1}} }{ \dotp{r_k}{r_k} }   \\
    p_{k+1} \leftarrow -r_{k+1} + \beta_{k+1} p_k\\
    k \leftarrow k+1
\end{cases}
\end{equation}
Pourquoi choisit-on cette m\'ethode? Qu'est-ce qui joue le r\^ole de $A$ et $b$ dans notre cas?
Attention, impl\'ementer cet algorithme de fa\c{c}on \`a pouvoir changer facilement $A$ et $b$.
\end{question}

\begin{question}
Utiliser l'agorithme pour \texttt{lena.jpg} \`a laquelle vous ajouterez un bruit gaussien, c'est \`a dire pour $ (i,j)\in [1,n]\times[1,m]$, on pose:
\begin{equation}
g_{i,j} = f_{i,j} + r_{i,j} 
\end{equation}
avec $r_{i,j}$ un tirage de variables gaussiennes ind\'ependantes de variance $\sigma$ et de moyenne nulle.
\end{question}

\begin{question}
Comparer la convergence de l'algorithme avec une descente de gradient \`a pas fixe.
\end{question}

\begin{question}
Discuter le r\'esultat en faisant varier les valeurs de $\beta$. Est-ce un probl\`eme de mod\'elisation ou un probl`eme num\'erique?
\end{question}


\section{Un autre mod\`ele}

Dans cette partie, on consid\`ere un autre mod\`ele qui fait intervenir la norme $L_1$ sur les matrices:
\begin{equation}
\| f \|_{1} = \sum_{i=1}^n \sum_{i=1}^m |f_{i,j}| \,.
\end{equation}

On souhaite remplacer la norme $L_2$ du gradient par la norme $L_1$ pr\'ec\'edemment d\'efinie.

\begin{question}
Proposer une version discr\`ete (c'est \`a dire sur les matrices) de la fonctionnelle $$ \frac12 \| u -  g\|^2_{L^2} + \frac{1}{2\beta}\| \nabla u \|_{L^1}\,.$$
\end{question}

\begin{question}
Pour $g=0$, $m=1$ et $n=2$, la fonctionelle est-elle (strictement) convexe comme fonction des coefficients de la matrice $u$?
\end{question}

\begin{question}
G\'en\'eraliser la question pr\'ec\'edente et discuter l'existence et l'unicit\'e de solutions.
\end{question}

\begin{question}
Pourquoi cette fonctionnelle ne rentre-t-elle pas dans le cadre d'applications des m\'ethodes \'etudi\'ees en cours?
\end{question}

\begin{question}
On propose de remplacer la norme $L^1$ par 
\begin{equation}\label{new}
\| f \|_{\varepsilon} = \sum_{i=1}^n \sum_{i=1}^m \sqrt{\varepsilon^2 + |f_{i,j}|^2} \,,
\end{equation}
pour un param\`etre $\varepsilon$ r\'eel.
\end{question}

\begin{question}
On propose de remplacer la norme $L^1$ par 
\begin{equation}
\| f \|_{\varepsilon} = \sum_{i=1}^n \sum_{i=1}^m \sqrt{\varepsilon^2 + |f_{i,j}|^2} \,,
\end{equation}
pour un param\`etre $\varepsilon$ r\'eel.
Que dire de la nouvelle fonctionnelle (r\'egularit\'e, convexit\'e, existence d'un minimum...)?
\end{question}

\begin{question}
Calculer le gradient de la fonctionnelle \eqref{new} et la hessienne.
\end{question}

\begin{question}
La m\'ethode de Newton utilise la direction de descente d\'efinie par:
$$ -[\nabla^2f(x)]^{-1}(\nabla f(x))\,.$$
Utiliser la m\'ethode du gradient conjugu\'e pour impl\'ementer la m\'ethode de Newton \`a pas fixe.
\end{question}

\begin{question}
Observer la vitesse de convergence et la comparer \`a une descente de gradient \`a pas fixe.
Comme le calcul de la direction de descente pour l'algorithme de Newton est relativement longue, \'etablir une comparaison prenant en compte le temps de calcul de chaque algorithme.
\end{question}

\begin{question}
Comparer les r\'esultats de ce nouveau mod\`ele avec le pr\'ec\'edent. Proposer une explication.
\end{question}

\end{document}




















\end{document}
