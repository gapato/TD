\documentclass[10pt,a4paper,fleqn]{report}
\usepackage{a4wide}
\usepackage{amsmath,amssymb}
\usepackage[french]{babel}
\usepackage[utf8]{inputenc}
\usepackage[T1]{fontenc}
\usepackage{lmodern}
\usepackage{oldstyle}
\usepackage{url}
\usepackage[pdfborder=(0 0 0)]{hyperref}
%%%%%%%%%%%%%%%%%%%%%%%%%%%%%%%%%%%%%%%%%%%%%%%%%%%%%%%%%%%%
\renewcommand{\thesection}{ \textbf{\Large \arabic{section}.}}
%%%%%%%%%%%%%%%%%%%%%%%%%%%%%%%%%%%%%%%%%%%%%%%%%%%%%%%%%%%\`{u}\`{u}
\setlength{\oddsidemargin}{1pt}
\setlength{\evensidemargin}{1pt}
\setlength{\textheight}{590pt}
\setlength{\textwidth}{470pt}

\makeatletter
\def\cleardoublepage{\clearpage\if@twoside\ifodd\c@page\else\hbox{}\thispagestyle{empty}\newpage\fi\fi}
\makeatother
\DeclareMathOperator*{\argmin}{arg\,min}
\newcommand{\matlab}{\textsc{Matlab}}

%%%%%%%%%%%%%%%%%%%%%%%%%%%%%%%%%%%%%%%%%%%%%%%%%%%%%%%%%%%%
\newcommand{\dotp}[2]{\langle #1 , #2\rangle}
\newcommand{\R}{\mathbb R}
\newcommand{\onit}{\begin{enumerate}}
\newcommand{\offit}{\end{enumerate}}
\newcommand{\grad}{\nabla}
\newcommand{\hess}{\nabla^2}
\newcommand{\on}{\begin{displaymath}}
\newcommand{\off}{\end{displaymath}}
\renewcommand{\P}{\mathcal P}
\newcommand{\push}[1]{\rule{0pt}{0pt}\hspace{#1pt}}
\renewcommand{\footnote}{\stepcounter{footnote}\raisebox{5 pt}{\scriptsize(\thefootnote)}\footnotetext}
\renewcommand{\tt}{\texttt}
\newtheorem{question}{Question}
%%%%%%%%%%%%%%%%%%%%%%%%%%%%%%%%%%%%%%%%%%%%%%%%%%%%%%%%%%%%

\begin{document}

\cleardoublepage

\noindent
Universit\'e Paris-Dauphine     \hfill      Optimisation num\'erique\\
D\'epartement MIDO              \hfill      François-Xavier Vialard\\
DUMI2E troisi\`eme ann\'ee      \hfill      ann\'ee \textos{2011}/\textos{2012}

\medskip

\hrule

\medskip

\begin{center}

\textbf{\huge S\'{e}ance de TD/TP n\textsuperscript{o}3}

\smallskip

\rule{10cm}{0.4pt}

\end{center}


\section*{Objectif}
On s'int\'eresse au probl\`eme de d\'ebruitage d'une image donn\'ee
par des m\'ethodes variationnelles. La donn\'ee d'une image
num\'erique est une matrice de taille $(n,n)$. On utilisera l'image
test classique en traitement d'image \texttt{lena.jpg} de taille
$(512,512)$ pour valider les algorithmes impl\'ement\'es. Les
m\'ethodes de d\'ebruitage reposent souvent sur des m\'ethodes
variationnelles qui passent par une repr\'esentation continue de
l'image. C'est \`a dire que les mod\`eles sont introduits sur des
espaces fonctionnels et une approximation discr\`ete est ensuite
utilis\'ee pour les tests num\'eriques.


\section{Norme $L^{2}$}

On rappelle qu'une image numérique au format RGB est composée de trois
matrices $M^{k}$, représentant les trois canaux de couleur rouge, vert
et bleu. Pour chacun de ces canaux, la valeur $M^{k}_{i,j}$ est un
entier entre 0 et 255 et représente l'intensité lumineuse (0 pour le
noir, 255 pour le blanc). Ainsi, une couleur violette (rouge + bleu)
très lumineuse au pixel $(i,j)$ sera encodée par $M^{1}(i,j) = 255,
M^{2}(i,j) = 0, M^{3}(i,j) = 255$.

\begin{question}
  Télécharger le fichier \verb+lena.jpg+ sur

  \url{http://www.ceremade.dauphine.fr/~levitt/optinum/}

  La charger sous la forme d'un tableau à trois dimensions $n \times n
  \times 3$ par la commande
  \footnote{Vous pouvez utiliser n'importe quelle image (carrée) de la même manière.}
  \begin{center}
      \verb+lena_rgb = imread('lena.jpg');+
  \end{center}
  Ce tableau est un tableau d'entiers non signés codés sur 8 bits, ce n'est pas adapté pour la manipulation
  d'image. On va donc le convertir en tableau de nombres à virgule flottante :
  \begin{center}
      \verb+lena_rgb = cast(lena_rgb, 'double');+
  \end{center}

  Par soucis de simplicité, on ne veut travailler sur un seul canal; pour ceci
  convertir l'image en niveaux de gris en faisant une moyenne sur les 3 canaux.
  \footnote{C'est une opération grossière qui ne prend pas en compte les particularités de la vision
  humaine : 2 lampes  bleue et verte de même puissance ne sont pas perçues comme ayant la même
  clarté. Pour plus d'info :

  \url{http://en.wikipedia.org/wiki/Grayscale}

  \url{http://fr.wikipedia.org/wiki/Efficacit\%C3\%A9_lumineuse_spectrale}
  }
  \begin{center}
      \verb!lena = (lena_rgb(:,:,1) + lena_rgb(:,:,2) + lena_rgb(:,:,3))/3!
  \end{center}
  On peut afficher l'image par la commande \verb+imagesc(lena);colormap('gray');+
\end{question}

\begin{question}
Impl\'ementer l'algorithme du gradient conjugu\'e pour r\'esoudre num\'eriquement le probl\`eme discretis\'e. On rappelle l'algorithme pour minimiser $\frac12 \dotp{Ax}{x}+\dotp{b}{x}$:

Initialisation de $k=0$,
$x_0 \in \R^{n^2}$,  $r_0 \leftarrow Ax_0 + b$ and $p_0 \leftarrow -r_0$ et it\'eration tant que $r_k\neq 0$ de
\begin{equation}
\begin{cases}
    \alpha_k \leftarrow \frac{\dotp{r_k}{r_k} }{ \dotp{p_k}{A p_k} } \\
    x_{k+1} \leftarrow x_k + \alpha_k p_k         \\
    r_{k+1} \leftarrow r_k + \alpha_k A p_k       \\
    \beta_{k+1} \leftarrow \frac{ \dotp{r_{k+1}}{r_{k+1}} }{ \dotp{r_k}{r_k} }   \\
    p_{k+1} \leftarrow -r_{k+1} + \beta_{k+1} p_k\\
    k \leftarrow k+1
\end{cases}
\end{equation}
Pourquoi choisit-on cette m\'ethode? Qu'est-ce qui joue le r\^ole de $A$ et $b$ dans notre cas?
Attention, impl\'ementer cet algorithme de fa\c{c}on \`a pouvoir changer facilement $A$ et $b$.
\end{question}

\begin{question}
Utiliser l'agorithme pour \texttt{lena.jpg} \`a laquelle vous ajouterez un bruit gaussien, c'est \`a dire pour $ (i,j)\in [1,n]\times[1,n]$, on pose:
\begin{equation}
g_{i,j} = f_{i,j} + r_{i,j}
\end{equation}
avec $r_{i,j}$ un tirage de variables gaussiennes ind\'ependantes de variance $\sigma$ et de moyenne nulle.
\end{question}

\begin{question}
Comparer la convergence de l'algorithme avec une descente de gradient \`a pas fixe.
\end{question}

\begin{question}
Que se passe-t-il pour des valeurs extrêmes de $\beta$ et comment l'expliquez-vous ?
\end{question}


\section{Norme $L^{1}$}

\begin{question}
La m\'ethode de Newton utilise la direction de descente d\'efinie par:
$$ -[\nabla^2f(x)]^{-1}(\nabla f(x))\,.$$
Utiliser la m\'ethode du gradient conjugu\'e pour impl\'ementer la m\'ethode de Newton \`a pas fixe.
\end{question}

\begin{question}
Observer la vitesse de convergence et la comparer \`a une descente de gradient \`a pas fixe.
Comme le calcul de la direction de descente pour l'algorithme de Newton est relativement longue, \'etablir une comparaison prenant en compte le temps de calcul de chaque algorithme.
\end{question}

\begin{question}
Comparer les r\'esultats de ce nouveau mod\`ele avec le pr\'ec\'edent. Proposer une explication.
\end{question}

\end{document}




















\end{document}
