\documentclass[10pt,a4paper,fleqn]{report}
\usepackage{a4wide}
\usepackage{amsmath,amssymb}
\usepackage[french]{babel}
\usepackage[utf8]{inputenc}
\usepackage[T1]{fontenc}
\usepackage{lmodern}
\usepackage{oldstyle}
\usepackage{parallel}
\usepackage{url}
%%%%%%%%%%%%%%%%%%%%%%%%%%%%%%%%%%%%%%%%%%%%%%%%%%%%%%%%%%%%
\renewcommand{\thesection}{ \textbf{\Large \arabic{section}.}}
%%%%%%%%%%%%%%%%%%%%%%%%%%%%%%%%%%%%%%%%%%%%%%%%%%%%%%%%%%%\`{u}\`{u}
\setlength{\oddsidemargin}{1pt}
\setlength{\evensidemargin}{1pt}
\setlength{\textheight}{590pt}
\setlength{\textwidth}{470pt}

\makeatletter
\def\cleardoublepage{\clearpage\if@twoside\ifodd\c@page\else\hbox{}\thispagestyle{empty}\newpage\fi\fi}
\makeatother

\newcommand{\matlab}{\textsc{Matlab}}

%%%%%%%%%%%%%%%%%%%%%%%%%%%%%%%%%%%%%%%%%%%%%%%%%%%%%%%%%%%%

\newcommand{\R}{\mathbb R}
\newcommand{\onit}{\begin{enumerate}}
\newcommand{\offit}{\end{enumerate}}
\newcommand{\grad}{\nabla}
\newcommand{\hess}{\nabla^2}
\newcommand{\on}{\begin{displaymath}}
\newcommand{\off}{\end{displaymath}}
\renewcommand{\P}{\mathcal P}
\newcommand{\push}[1]{\rule{0pt}{0pt}\hspace{#1pt}}
\renewcommand{\footnote}{\stepcounter{footnote}\raisebox{5 pt}{\scriptsize(\thefootnote)}\footnotetext}
\renewcommand{\tt}{\texttt}

%%%%%%%%%%%%%%%%%%%%%%%%%%%%%%%%%%%%%%%%%%%%%%%%%%%%%%%%%%%%

\begin{document}

\cleardoublepage

\begin{Parallel}{0.5\linewidth}{0.5\linewidth}
\ParallelLText{
\begin{flushleft}
Universit\'e Paris-Dauphine\\
D\'epartement MIDO\\
DUMI2E troisi\`eme ann\'ee
\end{flushleft}
}
\ParallelRText{
\begin{flushright}
Optimisation num\'erique\\
G\'erard Lebourg\\
ann\'ee \textos{2010}/\textos{2011}
\end{flushright}
}
\end{Parallel}

\medskip

\hrule

\medskip

\begin{center}

\textbf{\huge S\'{e}ance de Travaux Pratiques  n\textsuperscript{o}2}

\smallskip

\rule{10cm}{0.4pt}

\end{center}


\section*{Objectif} 
\'Etant donn\'e un panier de $N$ actifs dont le vecteur des rendements moyens $r$ et la matrice des variances-covariances $\Sigma$ sont connus, d\'eterminer la composition d'un portefeuille constitu\'e \`a partir de ces $N$ actifs assurant un rendement minimal anticip\'e $\rho$ en minimisant, pour $\rho$ et $c$ fix\'es, la fonction barri\`ere logarithmique~:
\[
f(x,\mu,\rho)=\frac{1}{2}\,x^T\Sigma\,x-\mu\,\sum_{i=1}^N\ln(x_i)-\mu\,\ln\left(r^Tx-\rho\right),
\]
sous la contrainte $\sum_{i=1}^N x_i=1$, avec $\mu$ un param\`etre r\'eel strictement positif destin\'e \`a tendre vers z\'ero.


\section{Pr\'{e}liminaires}
T\'{e}l\'{e}charger le fichier-archive \tt{optnum-tp2.tgz} \`{a} l'adresse suivante :

\centerline{\url{http://docs.ufrmd.dauphine.fr/optinum/tps/optinum-tp2.tgz}}
puis extraire les fichiers qui seront utilis\'{e}s durant la s\'{e}ance dans le r\'{e}pertoire
\tt{optinum-tp2}. Lancer alors \matlab\ et se placer dans le r\'{e}pertoire \tt{optinum-tp2}.

\section{\'Ecriture d'un oracle}

Le fichier \texttt{lbpfun.m} contenu dans le répertoire \tt{optinum-tp2} contient le modèle d'un oracle associé au critère du problème. Le code, reproduit ci-après,  déclare quatre variables \textit{globales} \tt{Sigma}, \tt r, \tt{mu}, et \tt{rho} destin\'ees \`a contenir respectivement la matrice des variances-covariances des rendements du panier d'actifs consid\'er\'e et le vecteur des rendements moyens, ainsi que les valeurs des paramètres$\mu$ et $\rho$~:

\begin{verbatim}
   lbpfun[f,g,H]=function(x)
   global Sigma r mu rho
   excess=r'*x-rho ;
   if all(x>0)&&(excess>0) % lbpfun defined at x
      f=0.5*x'*Sigma*x-mu*[sum(log(x))+log(excess)] ;
      if nargout>1
         g=
         if nargout>2
            H=
         end
      end
   else % lbpfun not defined at x
      f=inf ; g=r ; g(:)=NaN ; H=Sigma ; H(:,:)=NaN ;
   end
\end{verbatim}
\'Editer le fichier \tt{lbpfun.m}, compl\'eter le code manquant dans l'\'ecriture de la fonction (c'est-\`-dire les d\'efinitions de \tt{g} et de \tt{H}), et sauvegarder. 

\section{Acquisition des donn\'ees}

\onit
\item Entrer la commande \tt{global\  Sigma\ r\ mu\  rho\  pb}, qui déclare les variables globales \tt{Sigma},  \tt r, \tt{mu}, et~\tt{rho}, ainsi q'une variable~\tt{pb} destin\'ee \`a contenir une \textit{structure} rassemblant la totalit\'e des informations utiles sur le probl\`eme \`a  r\'esoudre, dont les champs seront~:

\onit
\item \tt{pb.crit}: une \og\textit{poign\'ee}\fg\ (handle) \tt{@lbpfun} sur la fonction \texttt{lbpfun},
\item \tt{pb.init}: l'initialisation \tt{x0} choisie pour  l'algorithme,
\item \texttt{pb.eqc}: une structure \tt{pb.eqc=\{pb.eqc.mat,pb.eqc.rhs\}} contenant la contrainte d'\'egalit\'e

\centerline{\tt{pb.eqc.mat*x=pb.eqc.rhs},}
soit encore, pour le probl\`eme trait\'e, simplement \tt{pb.eqc.mat=[1,1,...,1]}, et \tt{pb.eqc.rhs=1}.
\offit
\offit

Trois jeux de donn\'ees pourront être utilis\'es. Ils sont stock\'es dans un fichier \tt{data.m} contenant une \textit{structure} \tt{stocks}. Les trois jeux de donn\'ees \tt{stocks(pos)}, \tt{pos=1,2,3}, peuvent \^etre automatiquement charg\'es en utilisant la commande \tt{initialize(pos)} avec la valeur 1, 2, ou 3 de l'argument \tt{pos}. Cette commande affecte une valeur aux variables globales \tt{Sigma}, \tt{r}, \tt{mu},  \tt{rho}, et~\tt{pb}.  Les valeurs par d\'efaut de \tt{rho} et \tt{mu} sont \tt{rho=0} \footnote{0n cherche donc  \textit{a priori} le portefeuille de variance minimale sans contrainte de rendement.},  et \tt{mu=0.1}. Elles peuvent être modifiées à tout moment par une nouvelle affectation

\smallskip
 \onit 
\addtocounter{enumi}{1}
\item  Entrez par exemple la commande \texttt{initialize(1)} qui charge un premier jeu de données~: c'est un exemple \og jouet \fg\ de dimension trois.
\item Entrez alors les commandes \tt Sigma, puis~: \tt r -- sans la faire suivre d'un point-virgule -- pour \og voir \fg\ les valeurs des données correspondantes.
\offit

\section{\textsc{GUMP}: une bo\^ite \`a  outils commode}

Pour pouvoir comparer simplement l'application de diff\'erentes proc\'edures de recherche lin\'eaire et/ou de  diff\'erents choix de la direction de descente, on a cod\'e une proc\'edure \tt{gump} (\tt{GUMP} est l'acronyme pour \textbf{G}eneralized \textbf{U}nconstrained \textbf{M}inimization \textbf{P}rocedure) prenant pour argument deux structures: 
\smallskip

\begin{itemize}
\item \tt{pb}, qui contient le probl\`eme \`a  r\'esoudre,
\item \tt{algo}, qui contient la description de l'algorithme utilis\'e. 
\end{itemize}

\medskip

\onit
\item Pour comprendre l'utilisation de \tt{gump}, entrez d'abord la commande \tt{help gump}, et lire attentivement l'aide en ligne retourn\'ee par \matlab.
\item Entrez alors -- sans la faire suivre d'un point virgule -- la commande \tt{algo=gradopt},  qui charge dans la structure \tt{algo} l'algorithme de Gradient \`a pas optimal, puis la commande -- suivie cette fois d'un point-virgule -- \texttt{pts=gump(pb,algo,100)}. 
\item  R\'ecup\'erez le dernier point calcul\'e par l'algorithme en tapant la commande \texttt{x=pts(:,end)}, qui extrait la derni\`ere colonne de la matrice \texttt{pts}\footnote{Rappelez-vous la signification des deux points.}, et v\'erifiez \`a  l'aide des variables globales \texttt{r} et \texttt{Sigma} que le r\'esultat obtenu fournit un portefeuille de rendement anticip\'e sup\'erieur \`a $5$, et de risque (\'ecart-type ?) associ\'e inf\'erieur \`a $2$. 
\offit

Bien, vous avez d\'ecouvert le principe de diversification du risque qui a fait la fortune de H. Markowitz, laur\'eat du prix Nobel d'\'economie en 1990\footnote{En d\'emontrant  dans sa th\`ese, en 1952,  qu'il \'etait possible de r\'eduire le risque par diversification, il a fait entrer le loup des math\'ematiques dans la bergerie de la Finance. }. Passons maintenant aux choses s\'erieuses...

\bigskip

\section{Variations sur la recherche lin\'eaire}

\bigskip

\onit

\item Entrez la commande \texttt{initialize(3)} qui charge dans \texttt{pb} le probl\`eme de dimension 12 associ\'e \`a  \texttt{stocks(3)}. Les données correspondantes sont tirées de l'article :  \url{http://www.geo.ut.ee/nbc/paper/lapsina.htm} et
concernent douze actifs financiers, dont les actions les plus attractives des plus grandes compagnies -- les \textit{``Blue Chips''} -- du marché russe en 1997\footnote{Les estimations de la moyenne et de la matrice de dispersion du vecteur des rendements moyens de ces douze actifs sont obtenues à partir du relevé des rendements hebdomadaires de ces douze actions du 14.08.1996 au 13.08.1997.}. 
\item Entrez la commande \texttt{gump(pb,algo,100);}. Qu'observe-t-on? Expliquez.
\item Entrez alors la commande \verb+ algo.stepsize=@(phi)goldensearch(phi,10 ^-6);+ qui change la tol\'erance utilis\'ee par \texttt{goldensearch}. Tapez \texttt{algo} pour visualiser le changement effectu\'e dans la d\'efinition de l'algorithme, puis \texttt{pts=gump(pb,algo,1000);}. \textbf{Attention~:} n'oubliez-pas un z\'ero dans \tt{itermax} et n'oubliez pas le point-virgule !
\item Vous avez trouv\'e un r\'esultat raisonnable, mais pouvait-on faire mieux? Entrez maintenant la commande \texttt{algo.stepsize=@backtrack}. Puis essayez \`a  nouveau \texttt{gump(pb,algo,1000);}. Qu'en pensez-vous?
\item \`A l'\'et\'e 1997, \`a partir des estimations obtenues, quel portefeuille de variance minimale eussiez-vous conseill\'e de constituer avec les \textit{Blue Chips} du march\'e russe ? Donnez le rendement moyen et la variance de ce portefeuille.
\offit

\bigskip

Maintenant, c'est \`a vous de jouer.

\bigskip

\onit \setcounter{enumi}{5}
\item La proc\'edure \tt{backtrack} utilise l'interpolation quadratique de la fonction \tt{phi}  qui lui est pass\'ee pour argument pour trouver un pas \tt t v\'erifiant la \textit{condition d'Armijo}~: 
\tt{ phi(t)< phi(0)+ alpha*t*phi'(0)}. 
\onit
\item D\'eterminez l'\'equation de la parabole passant par les points \tt{(0,phi(0))} et \texttt{(t,phi(t))}, tangente en \tt{t=0} au graphe de \tt{phi}, puis l'abscisse du sommet de cette parabole en fonction de \tt{phi(t)}, \tt{phi(0)} et \tt{phi'(0)}.
\item \'Editez ensuite le fichier \tt{backtrack.m} et expliquez pourquoi la proc\'edure finit n\'ecessairement par trouver un pas \tt t v\'erifiant la condition d'Armijo (\textbf{indication}~: v\'erifier que l'abscisse du sommet de la parabole interpolante reste toujours strictement inf\'erieure \`a \tt{t/2(1-alpha))}.
\offit

\item Entrez la commande \tt{algo.stepsize=@Wolfe}, puis essayez \`a nouveau  \tt{pts=gump(pb,algo,2000);}. \'Editez le fichier \texttt{Wolfe.m} et expliquez pourquoi la proc\'edure finit toujours par trouver un pas \tt t v\'erifiant la r\`egle de Wolfe.

\item Pour finir, on sort le lapin du chapeau. Entrez la suite de commandes:

\hfill\parbox[t]{0.9 \textwidth}{
\tt{algo.memory=1;}\\
\tt{algo.searchdir=@BarzilaiBorwein;}\\
\tt{algo.stepsize=@backtrack;}\\
\tt{algo.stoppingtest=@gradtest;}\\
\tt{pts=gump(pb,algo,1000);}
}
\medskip

Qu'observe-t-on? \'Editez le fichier \tt{BarzilaiBorwein.m} et expliquez ce qui a chang\'e dans la construction de l'algorithme.
\offit

\bigskip

\section{Variations sur la direction de recherche}

\bigskip

\onit
\item \'Editez le fichier \tt{Cauchy.m} et expliquez le r\^ole de la matrice \texttt{trial.M} apparaissant dans le code de ce fichier (\textbf{indication}~: souvenez-vous que l'on minimise en fait $G(u)=F(x+N\,u)$).
\item Comment calculer alors la direction de recherche du gradient conjugu\'e ? (\textbf{indication}~: la formule g\'en\'erale d'actualisation est $\mathtt d_{k+1}=-g_{k+1}+\|g_{k+1}\|^2/\|g_k\|^2\,d_k$. Le probl\`eme est de prendre en compte l'existence \'eventuelle de contraintes d'\'egalit\'e lin\'eaires. Par cons\'equant, la matrice \tt{trial.M} doit intervenir quelquepart...). 
 \item Donnez le code d'une fonction \texttt{FR} prenant pour argument la structure \texttt{trial} et retournant la direction de  descente de Fletcher-Reeves (\textbf{attention}~: la premi\`ere direction de recherche doit \^etre la direction de Cauchy et \tt{trial.old.g}  vaut z\'ero \`a  l'initialisation.).
\item \'Ecrire alors une fonction \tt{gradconj} sur le mod\`ele de \tt{gradopt}. N'oubliez pas de mettre \tt{algo.memory} \`a  un. 
\item Entrez alors la suite de commandes:
\tt{ initialize(2);} \tt{algo=gradconj;} \tt{pts=gump(pb,algo,1000);},
puis corrigez la valeur de \texttt{algo.stepsize} en augmentant progressivement la pr\'ecision de \texttt{goldensearch} et recommencez. Qu'observe-t-on ? Expliquez.
\item Entrez la commande \tt{algo=quasinewt} -- sans la faire suivre d'un point-virgule -- pour voir comment est d\'efini l'algorithme, puis, à nouveau \tt{pts=gump(pb,algo,1000);} et comparez avec les résultats précédents.
\item Expliquez finalement comment définir une fonction \matlab\ \tt{newtopt} permettant de charger dans la structure \tt{algo} l'algorithme de Newton par la commande \tt{algo=newtopt}. Justifiez le contenu de chacun des champs de la structure \tt{algo} résultante. Détaillez en particulier le traitement de la contrainte d'égalité. Expérimentez.

\offit

\section{Conclusions}

\bigskip

Quelles sont vos conclusions? Comment choisiriez-vous de construire un algorithme  de sélection de portefeuille fondé sur la minimisation d'une barrière logarithmique? (Le résultat peut dépendre de la dimension du problème à résoudre). Justifiez vos choix.







\end{document}




















\end{document}
