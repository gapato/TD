\documentclass[10pt,a4paper,fleqn]{report}
\usepackage{a4wide}
\usepackage{amsmath,amssymb}
\usepackage[french]{babel}
\usepackage[utf8]{inputenc}
\usepackage[T1]{fontenc}
\usepackage{lmodern}
\usepackage{oldstyle}
\usepackage{parallel}
\usepackage{url}
%%%%%%%%%%%%%%%%%%%%%%%%%%%%%%%%%%%%%%%%%%%%%%%%%%%%%%%%%%%%
\renewcommand{\thesection}{ \textbf{\Large \arabic{section}.}}
%%%%%%%%%%%%%%%%%%%%%%%%%%%%%%%%%%%%%%%%%%%%%%%%%%%%%%%%%%%\`{u}\`{u}
\setlength{\oddsidemargin}{1pt}
\setlength{\evensidemargin}{1pt}
\setlength{\textheight}{590pt}
\setlength{\textwidth}{470pt}

\makeatletter
\def\cleardoublepage{\clearpage\if@twoside\ifodd\c@page\else\hbox{}\thispagestyle{empty}\newpage\fi\fi}
\makeatother

\newcommand{\matlab}{\textsc{Matlab}}

%%%%%%%%%%%%%%%%%%%%%%%%%%%%%%%%%%%%%%%%%%%%%%%%%%%%%%%%%%%%

\newcommand{\R}{\mathbb R}
\newcommand{\onit}{\begin{enumerate}}
\newcommand{\offit}{\end{enumerate}}
\newcommand{\grad}{\nabla}
\newcommand{\hess}{\nabla^2}
\newcommand{\on}{\begin{displaymath}}
\newcommand{\off}{\end{displaymath}}
\renewcommand{\P}{\mathcal P}
\newcommand{\push}[1]{\rule{0pt}{0pt}\hspace{#1pt}}
\renewcommand{\footnote}{\stepcounter{footnote}\raisebox{5 pt}{\scriptsize(\thefootnote)}\footnotetext}
\renewcommand{\tt}{\texttt}

%%%%%%%%%%%%%%%%%%%%%%%%%%%%%%%%%%%%%%%%%%%%%%%%%%%%%%%%%%%%

\begin{document}

\cleardoublepage

\begin{Parallel}{0.5\linewidth}{0.5\linewidth}
\ParallelLText{
\begin{flushleft}
Universit\'e Paris-Dauphine\\
D\'epartement MIDO\\
DUMI2E troisi\`eme ann\'ee
\end{flushleft}
}
\ParallelRText{
\begin{flushright}
Optimisation num\'erique\\
G\'erard Lebourg\\
ann\'ee \textos{2010}/\textos{2011}
\end{flushright}
}
\end{Parallel}

\medskip

\hrule

\medskip

\begin{center}

\textbf{\huge S\'{e}ance de Travaux Pratiques n\textsuperscript{o}3}

\smallskip

\rule{10cm}{0.4pt}

\end{center}


\section*{Objectif} 
Effectuer un zoom dans une image numérique \tt{img} en retournant~: 
$
Z= \mbox{arg} \min \left(\, S(Z)+ \mu\ E(Z)\,\right)
$
où~: 
\[
S(Z)=\frac{1}{2}\,\sum_{i=1}^M \sum_{j=1}^{N-1} |\,Z_i^j-Z_i^{j+1}\,|^2 +
\frac{1}{2}\,\sum_{i=1}^{M-1} \sum_{j=1}^N |\ Z_i^j-Z_{i+1}^j\,|^2
\]
est le \og Smoother \fg\ qui régularise l'image, $ E(Z)=\frac{1}{2}\,\|\ p(Z)-D\,\|^2$ mesure l'erreur commise en remplaçant la matrice $D$ -- obtenue en doublant grossièrement la taille de \tt{img} -- par $Z$, et $\mu$ est un paramètre permettant de régler l'arbitrage entre lissage et distorsion dans le rendu de l'image.


\section{Pr\'{e}liminaires}
T\'{e}l\'{e}charger le fichier-archive \tt{optinum-tp3.tgz} \`{a} l'adresse suivante :

\centerline{\url{http://docs.ufrmd.dauphine.fr/optinum/tps/optinum-tp3.tgz}}
puis extraire les fichiers qui seront utilis\'{e}s durant la s\'{e}ance dans le r\'{e}pertoire \tt{optinum-tp3}. Lancer alors \matlab\ et se placer dans le r\'{e}pertoire \tt{optinum-tp3}.


\section{Acquisition des donn\'ees}
\onit
\item Entrez la commande \tt{load zoom} chargeant l'environnement de travail \matlab\ \tt{zoom.mat}.
\item Entrez alors la commande \tt{whos} qui liste les variables définies dans cet environnement, la nature et la taille mémoire de chacune d'elles~:
\onit
\item[–] les variables \tt{visage} et \tt{singe} sont des $256\times 256$ matrices d'entiers codés sur huit bits (\tt{uint8}) correspondant à des images numérisées,
\item[–] la variable \tt{grayscale} est une matrice contenant une \og carte de couleur \fg\ (\tt{colormap}) permettant un affichage noir et blanc des images contenues dans \tt{visage} et \tt{singe},
\item[–] la variable \tt{mu} est destinée à contenir la valeur du paramètre $\mu$. Cette variable, déclarée comme \textit{globale}, pourra être modifiée à tout moment au moyen d'une simple commande d'affectation. 
\offit

\medskip

Le fichier \tt{optinum-tp3} contient le fichier \tt{draw.m} permettant un affichage des images utilisant la carte \tt{grayscale}. Vous pouvez ainsi:
\onit
\item[–] visualiser les images \tt{visage} ou \tt{singe} en entrant la commande \tt{draw(visage)}, ou \tt{draw(singe)}.
\item[–] convertir sous forme d'une matrice d'entiers une nouvelle image de votre choix en utilisant la commande \tt{imread(img.fmt)}, où \tt{img} est l'image à convertir, et \tt{fmt} le format \tt{jpeg}, \tt{png}, \tt{tiff}, etc...
\offit
\offit

\smallskip

En utilisant les résultats de ce tp, vous pourrez également extraire des morceaux de vos photos num\'eris\'ees noir et blanc personnelles et les agrandir...

\section{Extraction d'une partie de l'image}
\onit
\item Entrez par exemple la commande \tt{draw(singe)} et observez la manière dont \matlab\ indexe les pixels de l'image.
\item \'Ecrire dans un fichier le code d'une fonction \tt{cutout} prenant pour paramètre une matrice de pixels \tt{img} de format quelconque et deux couples d'entiers \tt{upperleft} et \tt{lowerright} extrayant la partie de \tt{img} dont les indices des coins supérieur gauche et inférieur droit sont respectivement contenus dans \tt{upperleft} et \tt{lowerright}. Inclure un test retournant un message d'erreur si \tt{any(size(img)<q)} ou \tt{any(q<p)}\footnote{\matlab\ se charge d'envoyer un message d'erreur si les composantes de \tt p et \tt q ne sont pas toutes des entiers strictement positifs, mais se contente de retourner une matrice vide lorsque la s\'election est invalide.}. Enregistrez ce fichier dans le répertoire \tt{optinum-tp3}.
\item Testez votre procédure en extrayant diverses parties des images \tt{visage} ou \tt{singe} que vous visualiserez en utilisant \tt{draw}.
\offit

\section{Principe de l'algorithme}

Le problème consiste à minimiser une fonction quadratique elliptique $F$ de la variable matricelle $Z$. La difficulté est la dimension de $Z$ 
($\simeq60000$ coefficients pour une image $256\times 256$) et la nature matricielle de la variable, qui rendent le système des conditions nécessaires -- un système linéaire de plusieurs milliers d'inconnues -- difficile à formuler et à résoudre. On utilisera ici l'algorithme du \textit{gradient conjugué}, parfaitement adapté à cette situation, qui, dans le cas d'un critère \textit{quadratique elliptique}, converge -- théoriquement -- en un nombre fini d'itérations.



\section{Utilisation de l'algorithme}

Le fichier \tt{zoom.m} du répertoire \tt{optinum-tp3} contient le code d'une fonction prenant pour arguments~:
\onit
\item[–] \tt{img}~: la matrice obtenue en extrayant une partie d'une image numérisée, par exemple~: une partie des images \tt{singe} ou \tt{visage},
\item[–] \tt{itermax}~: un entier indiquant le nombre d'itérations du gradient conjugué qui seront effectuées par l'algorithme.
\offit

La commande \tt{[Z,sqn]=zoom(img, itermax)} retourne~:
\onit
\item[–] \tt{Z}~: l'image \tt{img} doublée dans ses deux dimensions et lissée par \tt{itermax} passages dans la boucle du gradient conjugué. Cette variable est \textit{optionnelle}~: la commande \tt{zoom(img)} retourne simplement la matrice brute obtenue en doublant le nombre de pixels dans chaque dimension, sans qu'aucun lissage de l'image ne soit effectué.
\item[ –] \tt{sqn}~: (acronyme de \tt{SQuareNorm}) le carré de la norme du gradient du critère $F$ au point \tt{Z}.
\offit

\noindent La fonction \tt{zoom} déclare \tt{mu}comme une variable \textit{globale}, lui permettant d'utiliser la valeur de \tt{mu} prédéfinie dans l'environnement de travail \tt{zoom.mat}\footnote{En cas d'utilisation de l'algorithme en dehors de cet environnement, il sera nécessaire de déclarer au préalable \tt{mu} comme une variable globale et de lui affecter une valeur. Il est également préférable de charger la \og carte de couleurs \fg\ \tt{grayscale} contenue dans \tt{zoom.mat} et d'utiliser \tt{draw} pour visualiser correctement les images utilisées. \`A défaut, on utilisera la fonction \tt{imshow} contenue dans la \og boîte à outils\fg\ \tt{Image Processing Toolbox} de \matlab.}. Cette valeur peut être actualisée à tout moment par simple affectation.


\onit
\item Afficher l'une des deux images \tt{visage} ou \tt{singe} et choisir une zone significative pour le zoom (ex~: l'oeil du singe)
\item Extraire la zone choisie en utilisant la fonction \tt{cutout}, et la stocker dans une variable \tt{small}. Vérifier le résultat obtenu en entrant la commande \tt{draw(small)}.
\item Utiliser la commande \tt{raw=zoom(small);} pour stocker l'image brute de dimensions doubles dans une variable: \tt{raw}. Visualiser avec \tt{draw(raw)} l'effet de pixelisation.
\item Expérimenter l'action du lissage à l'aide de la commande \tt{Z=zoom(small,itermax);} en faisant varier \tt{itermax} et/ou la valeur du paramètre \tt{mu}. Observer seulement les valeurs du réel \tt{sqn} et visualiser, de temps en temps, le résultat correspondant en entrant \tt{draw(Z)}.

\noindent \textbf{Attention~: n'oubliez pas le point virgule si vous ne voulez pas être inondés de chiffres, car \tt{Z} est une grosse matrice !}
\offit

\medskip

Comment varie le résultat obtenu en fonction du choix du paramètre \tt{mu}? Quelle valeur de \tt{mu} vous parait fournir un arbitrage raisonnable entre lissage et distorsion? Si l'on devait modifier l'algorithme en incluant le test d'arrêt \tt{sqn<tol}, quelle valeur de l'argument \tt{tol} vous paraitrait judicieuse?

\section{Analyse de l'algorithme}

Souvenez vous que, lorsqu'on munit l'espace des $M\times N$ matrices réelles du produit scalaire 
\[
\left<A,B\right>= \mbox{tr}(A\star B')=\mbox{tr}(A'\star B)=\sum_{i=1}^M\sum_{j=1}^NA_i^j\,B_i^j,
\]
le gradient de $F$ peut être identifié à
\[
 \nabla F(Z)= A\star Z + Z\star B + \mu\star S\star Z\star T 
 \]
où $A$, $B$, $S$ et $T$ sont des matrices symétriques \textit{creuses}. L'algorithme construit ces matrices à l'aide des fonctions \tt{makeA}, et \tt{makeS} dont les codes sont contenus dans les fichiers éponymes du répertoire \tt{optinum-tp3}. Il fabrique une image brute en doublant les dimensions de l'image passée comme argument par appel à la fonction \tt{magnify}, et place le résultat dans une variable \tt{Z}, qui sera actualisée à chaque passage dans la boucle du gradient conjugué. Il crée et actualise deux autres variables \tt{G} et \tt{D}, destinées à contenir respectivement le gradient de $F$ au point courant \tt{Z}, et l'\textit{opposé}\footnote{Pour éviter des signes moins redondants.} de la direction du gradient conjugué en ce point. 

\onit
\item \'Editez le fichier \tt{zoom.m} contenu dans le répertoire \tt{optinum-tp3} et expliquez le code de la fonction auxiliaire \tt{mult}. 
\item Expliquez le calcul du pas optimal utilisé pour l'actualisation de \tt Z à chaque passage dans la boucle \tt{while}. Pourquoi est-il inutile d'utiliser une procédure de recherche linéaire ? Quelle propriété de la fonction $F$ permet de se passer ici d'une telle procédure ? 
\item Utilisation de la commande \tt{sparse}~:
\onit
\item \'Editez le fichier \tt{makeA.m} puis entrez la commande \tt{help sparse} pour comprendre la manière dont est codée la procédure.
\item Entrez la commande \tt{A=makeA(6)} -- sans point-virgule -- puis \tt{B=full(A)}, et observez la différence. Puis entrez les commandes \tt{A=makeA(60);} -- suivie cette fois du point virgule ! -- puis \tt{B=full(A);}, et enfin \tt{whos}, et comparez les tailles mémoire des variables \tt A et \tt B.
\item \'Editez de même le fichier \tt{makeS.m} et observez la manière dont est codée la procédure.
\item \'Editez alors le fichier \tt{magnify.m} et essayez d'expliquer la manière utilisée pour doubler la taille d'une image dans ses deux dimensions sans utiliser de boucle.
\offit
\item Usage des boucles \tt{for} et vectorisation en programmation \matlab~:
\onit
\item \'Editez le fichier \tt{gonfle.m} contenant le code d'une fonction éponyme qui, à l'instar de \tt{magnify}, double la taille de la matrice qui lui est passée pour argument, mais dont le code utilise cette fois deux boucles \tt{for}. Comparez avec \tt{magnify}. 
\item Entrez les commandes \tt{tic; gonfle(visage); toc}, puis \tt{tic; magnify(visage); toc}.
\item Recommencer en remplaçant l'argument \tt{visage} par \tt{Z}, après avoir entré la commande \tt{Z=magnify (visage)}.
\offit
Conclusion?
\offit





\end{document}




















\end{document}
