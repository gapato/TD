\documentclass[10pt,a4paper,fleqn]{report}
\usepackage{a4wide}
\usepackage{amsmath,amssymb}
\usepackage[french]{babel}
\usepackage[utf8]{inputenc}
\usepackage[T1]{fontenc}
\usepackage{lmodern}
\usepackage{oldstyle}
\usepackage{url}
%%%%%%%%%%%%%%%%%%%%%%%%%%%%%%%%%%%%%%%%%%%%%%%%%%%%%%%%%%%%
\usepackage{GLtd}
\renewcommand{\thesection}{ \textbf{\Large \arabic{section}.}}
%%%%%%%%%%%%%%%%%%%%%%%%%%%%%%%%%%%%%%%%%%%%%%%%%%%%%%%%%%%\`u\`u
\setlength{\oddsidemargin}{1pt}
\setlength{\evensidemargin}{1pt}
\setlength{\textheight}{590pt}
\setlength{\textwidth}{470pt}

\makeatletter
\def\cleardoublepage{\clearpage\if@twoside\ifodd\c@page\else\hbox{}\thispagestyle{empty}\newpage\fi\fi}
\makeatother

\newcommand{\matlab}{\textsc{Matlab}}

%%%%%%%%%%%%%%%%%%%%%%%%%%%%%%%%%%%%%%%%%%%%%%%%%%%%%%%%%%%%

\newcommand{\R}{\mathbb R}
\newcommand{\onit}{\begin{enumerate}}
\newcommand{\offit}{\end{enumerate}}
\newcommand{\grad}{\nabla}
\newcommand{\hess}{\nabla^2}
\newcommand{\on}{\begin{displaymath}}
\newcommand{\off}{\end{displaymath}}
\renewcommand{\P}{\mathcal P}
\newcommand{\push}[1]{\rule{0pt}{0pt}\hspace{#1pt}}
\renewcommand{\footnote}{\stepcounter{footnote}\raisebox{5 pt}{\scriptsize(\thefootnote)}\footnotetext}

%%%%%%%%%%%%%%%%%%%%%%%%%%%%%%%%%%%%%%%%%%%%%%%%%%%%%%%%%%%%

\begin{document}

\cleardoublepage

\noindent
Universit\'e Paris-Dauphine     \hfill      Optimisation num\'erique\\
Licence maths applis      \hfill      Ann\'ee \textos{2011}/\textos{2012}

\medskip

\hrule

\medskip



\begin{center}

\textbf{\huge TD 2}

\smallskip

\rule{10cm}{0.4pt}

\end{center}

\section{Introduction}
On se propose dans cette séquence de résoudre le problème de gestion
de portefeuille suivant. Soit une somme fixe d'argent $S$ à répartir
entre plusieurs investissements. On considère un modèle simplifié
d'investissement annuel où chaque actif est modélisé par son rendement
annuel par euro investi $X_{i}$. Ainsi, le rendement moyen $r_{i} =
E[X_{i}]$ représente le profit réalisé en investissant un euro sur
l'actif $i$. En contrepartie de ce profit, l'investisseur encourt un
risque, mesuré par la variance $\sigma_{i}^{2} = \text{Var}(X_{i})$.

Le problème est alors le suivant : comment investir son argent de
façon à s'assurer un rendement annuel moyen $\rho$ tout en minimisant
le risque encouru ?

Modéliser ce problème sous la forme d'un problème d'optimisation, en
explicitant les variables, la fonction objectif et les
contraintes. Peut-on considérer les actifs comme indépendants les uns
des autres ? De quelles caractéristiques des variables $X_{i}$ a-t-on
besoin dans la modélisation ?

\section{Étude théorique}
La modélisation a conduit à un problème de forme
\begin{align*}
  \text{min} \;\;\;\;&x^{T} A x\\
  \text{s.c.}\;\;\;\;&\begin{cases}
    &e^{T} x = S\\
  &r^{T} x \geq \rho\\
  &x \geq 0
  \end{cases}
\end{align*}
où $x, r \in \R^{n}$, $A \in M_{n}(\R)$, et $e = (1,1,\dots,1)$

\begin{exercice}
\begin{questions}
\item Pourquoi peut-on se restreindre au cas où la somme d'argent à
  investir $S$ vaut 1 ?
\item Quelles propriétés mathématiques la matrice $A$ vérifie-t-elle ?
\item Que se passe-t-il si $\rho$ est très grand ? Très petit ?
\item Quand l'ensemble admissible est non vide et que $A$ est
      d\'efinie positive, prouver l'existence et l'unicit\'e d'un minimiseur.
\item Résoudre le problème dans le cas $r \geq 0, \rho = 0, A =
  \begin{pmatrix}
    \sigma_{1}^{2}&0\\0&\sigma_{2}^{2}
  \end{pmatrix}$. Noter le \textit{principe de diversification des
    actifs} : $x_{i}$ n'est pas nul, même s'il est associé à un risque
  très élevé.
\end{questions}
\end{exercice}

\begin{exercice}[(portefeuille de variance minimale)]
  On veut savoir quand la contrainte $r^{T} x \geq \rho$ est
  active. Soit $$\hat x= \mathrm{ArgMin} \{\ x^{T} A x\ |\ e^{T} x=1, \
  x\geq 0\ \}$$ le portefeuille de variance minimale.
\begin{questions}
\item Prouver que, si $A^{-1}\,e\geq 0$, $\hat x=(e^{T} A^{-1} e)^{-1}\,A^{-1}e$.
\item Montrer que cette condition est, en particulier, v\'erifi\'ee si $A$ est diagonale.
% \item On dispose de trois actifs de rendements non corr\'el\'es, de moyennes $r_1=0.05$, $r_2=0.10$, $r_3=0.15$, et de variances $\sigma_1^2=0.0004$, $\sigma_2^2=0.0064$, et $\sigma_3^2=0.01$. Quel est le portefeuille de variance minimale? Quel est le rendement esp\'er\'e de ce portefeuille ?
\item Prouver que, si $\rho$ est strictement sup\'erieur au rendement du portefeuille de variance minimale, la contrainte $r^{T} x \geq \rho$ est n\'ecessairement active \`a la solution du probl\`eme.
\end{questions}
\end{exercice}

\begin{exercice}[(projection)]
  On se place maintenant dans le cas où la contrainte $r^{T} x \geq
  \rho$ est active.  Pour $x$ donn\'e on veut calculer sa projection
  orthogonale $p_{F}(x)$ sur l'espace affine des contraintes
  \[F = \{x \; | \; r^{T} x = \rho, e^{T} x = 1\}.\]
  
  Ecrire $p_F(x)$ comme solution d'un problème d'optimisation, et le
  r\'esoudre.  Dans quel cas peut-on projeter sur les deux contraintes
  successivement, au lieu de simultanément ?
\end{exercice}

\section{Implémentation Matlab}
\begin{exercice}
  On se propose de résoudre le problème
\begin{align*}
  \text{min} \;\;\;\;&x^{T} A x\\
  \text{s.c.}\;\;\;\;&\begin{cases}
    &e^{T} x = 1\\
  &r^{T} x = \rho\\
  &x \geq 0
  \end{cases}
\end{align*}

par la méthode du gradient projeté. On rappelle que cette itération
s'écrit $x_{n+1} = p_{C}(x_{n} - t \nabla F(x_{n}))$, où $p_{C}$ est
la projection orthogonale sur l'espace des contraintes $C$.

Pour toutes les fonctions à coder,
on déclarera les variables nécessaires ($A, r, \rho, e$) comme
globales.
\begin{questions}
\item Écrire un oracle \verb+oracle_tp2+ qui renvoie la valeur et le
  gradient de la fonction objectif.
\item Écrire une fonction \verb+proj_tp2+ qui projette $x$ sur l'ensemble
  des contraintes. Plutôt que d'effectuer la projection complète,
  difficile à calculer, on pourra se contenter de projeter sur les
  deux contraintes d'égalité, puis de projeter sur les contraintes
  d'inégalité.
\end{questions}

\end{exercice}

\section{Compléments}
\begin{exercice}[(m\'ethode du gradient r\'eduit)]
On souhaite minimiser la fonction
\on F(x)=x^T A x \off
sous les contraintes $e^{T} x=1, r^{T}x = \rho$.
\begin{questions}
\item Comment construire une matrice $P$ dont les colonnes forment une
  base du noyau de l'application lin\'eaire $\R^{n} \to \R^{2}$ qui
  \`a $x$ associe $e^{T} x$ et $r^{T} x$ ?
\item Prouver alors que, si $x_{0}$ est un point admissible, minimiser
  $F(x)$ sous les contraintes \'equivaut \`a minimiser $G(u)=F(x_0+P
  u)$ \it sans \rm contrainte.
\item Calculer $\grad G(u)$ et $\hess G(u)$ en fonction de $\grad
  F(x)$ et $\hess F(x)$, o\`u $x=x_0 + P \star u$.
\end{questions}
\end{exercice}

\newpage

\begin{center}

\textbf{\huge TP 2}

\smallskip

\rule{10cm}{0.4pt}

\end{center}

On résoud le problème
\begin{align*}
  \text{min} \;\;\;\;&x^{T} A x\\
  \text{s.c.}\;\;\;\;&\begin{cases}
    &e^{T} x = 1\\
  &r^{T} x = \rho\\
  &x \geq 0
  \end{cases}
\end{align*}

par la méthode du gradient projeté à pas fixe. Les fichiers à
télécharger sont disponibles sur

\verb+http://ceremade.dauphine.fr/~levitt/optinum/+

\begin{enumerate}
\item Coder les fonctions de l'exercice 4, et les tester.
\item Télécharger le fichier \verb+gradfix.m+, et l'utiliser comme
  modèle pour une fonction \verb+gradproj+ qui accepte en paramètre
  une fonction \verb+proj+ chargée de réaliser la projection. Que se
  passerait-il si on avait utilisé la norme du gradient comme critère
  d'arrêt ?
\item Comment choisir le pas de la méthode de gradient ? Le point
  initial ?
\item Télécharger le fichier \verb+data.m+, et tester votre algorithme
  avec les jeux de données fournis.
\item Tracer la fonction de recherche linéaire $\phi(t) = F(x-t\nabla
  F(x))$ pour différentes valeurs de $x$, et pour différents jeux de
  données. Que constatez-vous ? Expliquez.
\item Modifier la procédure \verb+gradproj+ pour qu'elle retourne la
  liste des itérés $x_{n}$. Tracer la courbe de convergence $||x_{n+1}
  - x_{n}||$ en fonction de $n$.
\item L'analyse théorique de l'algorithme de gradient mène à un
  résultat de convergence linéaire du type
  \begin{align*}
    ||x_{n} - x^{*}|| \approx_{n\to \infty} K \nu^{n}
  \end{align*}
  où $K > 0$, et $\nu < 1$ est la constante de convergence.

  Combien d'itérations faut-il pour obtenir une précision de
  $10^{-16}$ sur $x$ ?
  
  Prouver que
  \begin{align*}
    ||x_{n+1} - x_{n}|| \approx_{n \to \infty} K \nu^{n}
  \end{align*}

  Montrer qu'un tracé en échelle semi-logarithmique permet de vérifier
  visuellement cet équivalent. Se documenter sur les possibilités de
  régression linéaire en Matlab pour obtenir numériquement la constante
  $\nu$.
\item Comment la constante de convergence $\nu$ varie-t-elle en
  fonction du pas $t$ ?
\end{enumerate}
\end{document}
