\documentclass[10pt]{article}
% Francais UTF8
\usepackage[french]{babel}
\usepackage[utf8]{inputenc}
\usepackage[T1]{fontenc}
\usepackage{lmodern}
\usepackage{url}
\usepackage{graphicx}
\usepackage[small]{caption}
% Maths de l'AMS. Dispo partout
\usepackage{amsmath, amssymb}
%\usepackage{fullpage}
\usepackage[margin=2cm]{geometry}
\newcommand{\R}{\mathbb{R}}
\newcommand{\Z}{\mathbb{Z}}

\newtheorem{theorem}{Théorème}
\newtheorem{definition}{Définition}

\newcounter{numeroexo}
\newcommand{\exercice}{\par\noindent\stepcounter{numeroexo}
	\hspace{-.25cm}\fbox{\textbf{Exercice \arabic{numeroexo}}}\quad}

\begin{document}
\title{Algèbre linéaire 3 : exercices supplémentaires}
\maketitle
\section{Espaces vectoriels}
\exercice Montrer que toute matrice de rang $1$ s'écrit sous la forme
$M = x y^T$ avec $(x,y)\in\mathbb{R}^n\times\mathbb{R}^n$.

\exercice (Opérations sur des espaces vectoriels) Soit $E$ un espace
vectoriel, $S_1$ et $S_2$ deux sous-espaces. Dites si les espaces
suivants sont des espaces vectoriels. En supposant que $E$ est de
dimension finie, donner sa dimension, et un procédé de construction
d'une base à partir de celles de $S_1$ et $S_2$.
\begin{align*}
  E_1 &= S_1 \cap S_2\\
  E_2 &= S_1 \cup S_2\\
  E_3 &= S_1 + S_2 = \{x + y, x \in S_1, y \in S_2\}\\
  E_4 &= S_1 \times S_2 = \{(x, y), x \in S_1, y \in S_2\}\\
\end{align*}

Pour $E_4$, qui n'est pas un sous-ensemble de $E$, définir les
opérations d'addition et de multiplication pour un scalaire.

\exercice (Interpolation) On considère le problème d'interpolation
polynomiale : étant donné des réels $(x_i)_{1 \leq i \leq n}$ et
$(y_i)_{1 \leq i \leq n}$, on cherche un polynome $P$ de degré $n-1$
qui interpole le jeu de valeurs, c'est-à-dire tel que
\begin{align*}
  P(x_i) = y_i, 1 \leq i \leq n
\end{align*}
\begin{itemize}
	\item Réécrire ce problème sous la forme d'une équation matricielle $V A =
Y$, o\`u $A$ est le vecteur des coefficients du polynome $P$ : $P(X) =
a_0 + a_1 X + \dots + a_{n-1} X^{n-1}$.
La matrice $V$ est appelée
matrice de Vandermonde, et le calcul de son déterminant est un
exercice classique en théorie des
déterminants\footnote{\url{http://fr.wikipedia.org/wiki/Matrice_de_Vandermonde}}.
	\item Démontrer que si les $x_i$ sont tous distincts, la matrice $V$ est inversible, et ainsi
		prouver l'existence et l'unicité du problème d'interpolation polynomiale.
		\emph{Indice : on pourra montrer que si $V A = 0$, alors $P$ est nul.}
	\item Que se passe-t-il si on considère des polynomes
		de degré supérieur ou inférieur à $n-1$ ?
\end{itemize}

\exercice (Polynomes trigonométriques)
\begin{itemize}
	\item Montrer que $P$, l'espace des
fonctions continues périodiques de période $2\pi$, est un espace
vectoriel. Quelle est sa dimension ?
	\item Montrer que la famille $(\sin(x),
\sin(2x), \sin(3x))$ est une sous- famille libre de $P$. Est-elle
génératrice ?
\end{itemize}
On peut montrer qu'en un certain sens, la famille
$\{\sin(nx), \cos(nx), n \in \Z)$ est une ``base de dimension
infinie'' de $P$. En d'autres termes, pour tout $f$ dans $P$, on a
\begin{equation*}
  f(x) = \sum_{n \in \Z} a_n \cos(nx) + b_n \sin(nx).
\end{equation*}
C'est la décomposition en série de Fourier, qui a d'innombrables
applications, en mathématiques pures et appliquées
(JPEG\footnote{Détails sur \url{http://fr.wikipedia.org/wiki/JPEG}}, MP3 ... à
chaque fois le principe est de garder un nombre fini de coefficients
$a_n$ et $b_n$ pour reconstituer le signal à partir d'un nombre réduit
de variables)

\exercice (Équivalence de normes) Prouver les inégalités suivantes sur $\R^N$. À chaque fois,
donner un cas d'égalité, ce qui prouve que les inégalités sont
optimales (on ne peut pas remplacer les constantes par des constantes
plus faibles)
\begin{align*}
  ||x||_1 &\leq \sqrt N ||x||_2 \text{ indice : utiliser Cauchy-Schwartz}\\
  ||x||_2 &\leq \sqrt N ||x||_\infty\\
  ||x||_\infty &\leq N ||x||_1
\end{align*}

Ceci prouve que ces trois normes sont équivalentes (c'est-à-dire
définissent la même topologie : même notions de continuité, de
limites, etc.). Ce sont des cas particuliers du théorème fondamental :
toutes les normes sont équivalentes en dimension finie.

\exercice Soit $f$ une application linéaire sur $E$ de dimension
finie. On définit
\begin{align*}
  N(f) = \sup_{||x|| = 1} {||f(x)||}.
\end{align*}

Montrer que $N$ est bien définie, puis que c'est une norme sur
l'espace $\text{End}(E)$ des endomorphismes de $E$.

\section{Produit scalaire}

\exercice (Polynomes orthogonaux) On considère l'espace $E_n$ des
polynomes de degré $n$. On considère la fonction
\begin{align*}
  \langle P, Q\rangle &= \int_0^1 P Q.
\end{align*}

Montrer que c'est un produit scalaire. Donner une base orthonormée de
$E_0, E_1, E_2$. Indice : on pourra orthonormaliser la base canonique
de l'espace des polynomes grâce au procédé de Gram-Schmidt. La base
ainsi obtenue forme les polynomes de Legendre. Il existe toute une
ménagerie de polynomes orthogonaux pour différents produits
scalaires\footnote{\url{http://fr.wikipedia.org/wiki/Polynomes_orthogonaux#Tableau_des_polynomes_orthogonaux_classiques}},
avec des applications en physique, théorie de l'approximation, etc.


\exercice (Polynomes trigonométriques, le retour) On considère la
famille infinie $(\cos(nx), \sin(nx))$ d'éléments de $P$, espace des
fonctions continues de période $2\pi$, muni du produit scalaire
(vérifier que c'en est bien un)
\begin{align*}
  \langle f,g \rangle &= \int_0^{2\pi} f g.
\end{align*}
Montrer que cette famille est orthogonale. Comment pourrait-on obtenir une
famille orthonormale ?

\section{Projecteurs}
\exercice Montrer que $M = x x^T$ représente la projection orthogonale
sur le sous-espace engendré par $x$. En s'inspirant de cette forme,
donner l'expression de la projection orthogonale sur un sous-espace.

\exercice (Moindres carrés) Reformuler les résultats du cours en
utilisant le langage matriciel, sous la forme: trouver $x \in \R^n$
qui minimise
\begin{align*}
  F(x) = ||A x - b||_2^2,
\end{align*}
o\`u $b \in \R^m$ et $A \in M_{n,m}(\R)$ sont fixés. $n$ est le nombre
de paramètres, $m$ le nombre d'observations, $m \gg n$.

On rappelle les conditions nécessaires pour que $x$ soit le minimum de
$F$:
\begin{align*}
  \forall y \in \R^n, \frac{d}{d \varepsilon} F(x+\varepsilon y) = 0
\end{align*}
(ou, autrement dit, $\nabla F(x) = 0$). En déduire que, pour le
minimum, on a
\begin{align*}
  A^T A x = A^T b
\end{align*}

Ces équations expriment les conditions d'orthogonalité vues en cours
(le vérifier sur le cas bien connu de la droite des moindres carrés)
\end{document}
